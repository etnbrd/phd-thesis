# Explanation of the concept

## Turn-based programming.







(see presentation on Dues)
-> single-thread, no wait, no block and so on
Shared heap -> no mutex, no synchronization, so it is good scalability


Turn-based programming is an event-loop.
It is the execution of queued events one after the other.
An event is the association of a callback and a message.
The callback is a small Javascript Program, designed to process the message.
During its turn, the callback executes, and can queue events : that is register callback to be executed during a next turn.
TODO what I mean exactly by queue events ? -> the distinction between the asynchronous operation, and the resulting event.

## Pipeline

So a callback sends messages to other callbacks.
-> It is exactly like a pipeline.
However, all the callbacks share the same heap.
So it is not possible to distribute the different callbacks without synchronization of this heap, or splitting the heap for each callback.
TODO state VERY clearly this problem, it is at the core of my thesis.

So, how to split the heap so that each callback has its own exclusive heap ?

## Propagation of variables.

Javascript is lexically scoped, therefore we can identify the the scope of variable statically.
(At the exception of eval and with : with is forbidden from strict mode, so that is not a bigdeal, howether, eval is sometimes used in smart ways, but most of the time it is monomorphic (I don't exactly know what that means, I heard from Floreat, it must be something related to PL community)).

### Scope identification

The compiler identifies the variables shared by multiple callbacks from their scope.
TODO explain this in depth.
Function scope, closures, and so on ...

### Scope leaking

Javascript uses a pass-by-sharing paradigm.
That means that sometimes the argument of a call are passed by value, sometimes by reference (atomic data type (number, string, bool) -> by value, complex data type (objects) -> by reference).
That means that the modification of a local variable can affect variable in seemingly unrelated scopes.
It seems that the points-to analysis is what is used to find stuffs like that (side-effects ?).

### Propagation of execution and variables

The execution progress downstream, following the message stream.
TODO state very clearly this proposition, it is the second core of my thesis (and I love the idea, it relates directly to reality, graivity, and the fabric of the universe <3).

Because the propagation of the modification is not instantaneous, going back upstream is like going backward in time : it is impossible.
Therefore, a variable cannot be read upstream a write.
And it cannot be write downstream either.

In other words, only one callback can write on a variable -> seems obvious from previous sections.


In promises, because the heap is not shared, things are less restrictive.
Multiple stages can read and write the same variable, because the propagation of modification is instantaneous, due to the shared heap.