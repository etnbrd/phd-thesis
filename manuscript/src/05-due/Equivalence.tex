\section{Equivalence} \label{section:equivalence}

\nt{TODO this title is not clear}

We present the transformation from a nested imbrication of continuations into a chain of Dues.
% The transformation must preserve the semantic.
We explain the three limitations imposed by our compiler for this transformation to preserve the semantic.
They preserve the execution order, the execution linearity and the scopes of the variables used in the operations.

\subsection{Execution order}

Our compiler spots function calls with a continuation, which are similar to the abstraction in (\ref{eq:order:source}).
It wraps the function $fn$ into the function $fn_\textbf{due}$ to return a Due.
And it relocates the continuation in a call to the method $\textbf{then}$, which references the Due previously returned.
The result should be similar to (\ref{eq:order:target}).
The differences are highlighted in bold font.
\begin{equation} \label{eq:order:source}
fn([arguments], continuation)
\end{equation}
\begin{equation} \label{eq:order:target}
fn_\textbf{due}([arguments])\textbf{.then}(continuation)
\end{equation}

The execution order is different whether $continuation$ is called synchronously, or asynchronously.
If $fn$ is synchronous, it calls the $continuation$ within its execution.
It might execute $statements$ after executing $continuation$, before returning.
If $fn$ is asynchronous, the continuation is called after the end of the current execution, after $fn$.
The transformation erases this difference in the execution order.
In both cases, the transformation relocates the execution of $continuation$ after the execution of $fn$.
For synchronous $fn$, the execution order changes ; the execution of $statements$ at the end of $fn$ and the continuation switch.
The latter must be asynchronous to preserve the execution order.

\subsection{Execution linearity}

Our compiler transforms a nested imbrication of continuations, which is similar to the abstraction in (\ref{eq:state:source}) into a flatten chain of calls encapsulating them, like in (\ref{eq:state:target}).
\begin{align} \label{eq:state:source}
&fn1([arguments], cont1 \{\nonumber\\
&\qquad  declare ~ variable \leftarrow result\nonumber\\
&\qquad  fn2([arguments], cont2 \{\nonumber\\
&\qquad\qquad    print ~ variable\nonumber\\
&\qquad  \})\nonumber\\
&\})
\end{align}
\begin{align} \label{eq:state:target}
&\textbf{declare variable}\nonumber\\
&fn1_\textbf{due}([arguments])\nonumber\\
&\textbf{.then}(cont1\{\nonumber\\
&\qquad  variable \leftarrow result\nonumber\\
&\qquad  fn2_\textbf{due}([arguments])\nonumber\\
&\})\nonumber\\
&\textbf{.then}(cont2\{\nonumber\\
&\qquad  print ~ variable\nonumber\\
&\})
\end{align}

An imbrication of continuations must not contain any loop, nor function definition that is not a continuation.
Both modify the linearity of the execution flow which is required for the equivalence to keep the semantic.
A call nested inside a loop returns multiple Dues, while only one is returned to continue the chain.
A function definition breaks the execution linearity.
It prevent the nested call to return the Due expected to continue the chain.
% And a call nested inside a function definition is unable to return the Due to continue the chain.
On the other hand, conditional branching leaves the execution linearity and the semantic intact.
If the nested asynchronous function is not called, the execution of the chain stops as expected.

% We demonstrate the equivalence with a sequence of two continuations.
% The equivalence is sound for a sequence of \textit{n} continuations.

\subsection{Variable scope}

In (\ref{eq:state:source}), the definitions of $cont1$ and $cont2$ are overlapping.
The $variable$ declared in $cont1$ is accessible in $cont2$ to be printed.
In (\ref{eq:state:target}), however, definitions of $cont1$ and $cont2$ are not overlapping, they are siblings.
The $variable$ is not accessible to $cont2$.
It must be relocated in a parent function to be accessible by both $cont1$ and $cont2$.
To detect such variables, the compiler must infer their scope statically.
Languages with a lexical scope define the scope of a variable statically.
Most imperative languages present a lexical scope, like C/C++, Python, Ruby or Java.
The subset of Javascript excluding the built-in functions \texttt{with} and \texttt{eval} is also lexically scoped.
To compile Javascript, the compiler must exclude programs using these two statements.