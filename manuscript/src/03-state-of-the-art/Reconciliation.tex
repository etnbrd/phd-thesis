\section{Reconciliations}




\subsection{Design patterns}

\subsubsection{Algorithmic Skeletons}
\cite{McCool2010} Mc Cool, Structured Parallel Programmin with Deterministic Patterns
\textit{The general idea is that specific combinations of computation and data access recur in many different algorithms.}

\subsubsection{Accelerators} -> Annotations
CUDA, OpenCL


Difference between skeletons and accelerators ?

Imperative
  Piccolo      Parallel in-memory \cite{Power2010}
  CIEL         Stateless dataflow \cite{Murray2011}
  Statdeful Dataflow Graph (SDG)          Stateful dataflow  \cite{Fernandez2014a}



\subsubsection{Lock-free Algorithm}

Lock-free algorithm are highly concurrent, as they can be replicated, however, they are limited, and really hard to develop.
\url{https://en.wikipedia.org/wiki/Non-blocking_algorithm}

\subsubsection{Microservices \& SOA}

Microservices are in the reconciliation category, I think.
It is an attempt at reconciling the two organization. 
They advocate that software developers can manage the two organizations at a sufficiently fine level.
However, it doesn't support growth as well as sequential programming.



\subsection{Compilation}

Parnas already advocated conciling the two methods in its Information Hiding paper.
Using an assembler to transform the high-level, development, vision into the low-level, execution, vision.


\subsubsection{Cyclic parallelism}
Data parallelism / Vectorization

\subsubsection{Pipeline parallelism}
Task parallelism / Pipeline parallelism


An Overview of the SUIF Compiler for Scalable Parallel Machines. \cite{Amarasinghe1995}


Interesting articles :

http://comjnl.oxfordjournals.org/content/early/2015/09/15/comjnl.bxv077.abstract


-> THIS, to read
Load balanced pipeline parallelism \cite{Kamruzzaman2013}


\subsubsection{Static analysis}

Javascript static analysis here

\endinput

There are attempts at conciliating the two approaches.

Without a transformation process :
Erlang - Jay Nelson, Structured Programming Using Processes









Continuations and coroutines \cite{Haynes1984}
-> THIS

Parallel closures, a new twist on an old idea \cite{Matsakis2012a}

Making state explicit ... \cite{Fernandez2014a}

http://2015.splashcon.org/event/splash2015-splash-i-lindsey-kuper-talk


Continuation of work on SEDA \cite{Salmito2014}