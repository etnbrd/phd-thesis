
\section{Scalability}
  \subsection{Theories}
    \subsubsection{Linear Scalability}
    \subsubsection{Limited Scalability}
    \subsubsection{Negative Scalability}
      \comment{Conclusion : scalability = concurrency + not sharing the resources that grows with the scale}
  \subsection{Scalability outside computer science (only if I have time)}
      \comment{
        If I have time, I would like to try to explain why scalability is at the core of material engagement and information theory,
        and is at the core of our universe : the propagation of Gravity wave is an example : it is impossible to scale
        }

%-----------------------------------------------------------------------------%
                                    \endinput
%-----------------------------------------------------------------------------%


% TODO introduction to this chapter


We define a web service as a computer program whose main interface is based on web protocols, such as HTTP.
Such a service uses resources allocated on a network of computers.
Scalability defines the ability of the service to use a certain quantity of resource to meet a desired performance.
We call system the association of the computer program and the available resources. 
The performance of this system is measured by its latency and throughput.

\subsection{Latency and throughput}

Latency is the time elapsed between the reception of a request, and the sent of the reply.
It includes the time waiting for resources to be free to process the request, and the time to process the request.

Throughput is the number of requests processed by the system by unit of time.

Latency and throughput are linked in a certain way.
If a modification of the web service reduces its mean latency to a half, then the throughput doubles immediately.
It takes half the time to process a request, therefore, the service can process more requests in the same time.
However, if throughput augment, the latency doesn't necessarily decrease.

\subsection{Scalability granularity}

We define a computer program as a set of operations.
In the case of a web service, these operations can be directly requested by the user through the interface.
An operation can cause any other operation to execute.

Because both the resources used and the operations executed are discrete : not infinitely divisible, scalability is inherently discrete.


% TODO scalability granularity ? 
Scalability granularity is the increment of resources.
How the input data can be split up ?
How the program can be deployed on many machines ?






We call system the association of the computer program with the resources


\subsection{Horizontal and vertical scaling}

There is two ways to augment the resources of the system.
Enhance the nodes in the computer network - vertical scaling.
Or add more nodes to the computer network - horizontal scaling.


% TODO how to apply these theory to highly concurrent servers ?
% How does it modify the theories ?

There are three theories, from the most restrictive, to the most general.


% TODO define scalability
Scalability is the property of a computer program to occupy available resources to meet a needed performance.
EIther in Latency, or in throughput.







\subsection{Linear scalability}

Clements et. al. \cite{Clements2013a} prove that a computer program scale linearly if all its operations are commutative.
% Commutative is not parallel. How to go from commutative to parallel ?
Two operations are said to be commutative if they can be executed in any orders, and the same initial state will result in the same final state.
Commutativity implies the two operations to be memory-conflict free, or independent, which is equivalent to say that they can be executed in parallel.

Therefore, to achieve linear scalability, a computer program must be composed of a set of operations that commutes.
Thus, all the operations are parallel, they can be executed simultaneously, on any number of machines as required.

The size of the operations sets the scalability granularity.

However, commutativity is not achievable in real applications.
Even sv6, the operating system resulting from the work on commutative scalability only has 99\% commutativity.
For real application, in the best case, the granularity is coarse, in the worst case, there is no possible commutativity because of shared resources (like a product inventory, or a friend graph).

% TODO continue

\subsection{Limited scalability}

Amdahl introduced in 1967 a law to predict the limitation of speedup a computer program can achieve if a fraction of its code is sequential.
Amdahl worked at increasing the speed of computer clock, while the scientific community was working on improving parallelism of computing machines.

In a set of operations, even if one is non-commutative, it cannot be executed in parallel of any others, the scalability is limited by this operation.

There is a difference if the operation is non-commutative with itself, or only with others.
In the first case, it impose a queuing, while in the second case, it only increase the granularity : you can regroup the non-commutative operation with its subsequents, and form a bigger commutative operation.

% TODO continue

\subsection{Negative scalability}

Gunther generalized Amdahl's law into the Universal Scalability Law.
It includes the parallelization of non-independent operations with the use of synchronization.

It models the negative return on scalability from sharing resources observed in many real world applications.

% TODO continue by explaining the different area of scalability.



\subsection{Eventual Consistency}

To overpass the scalability limits set by the previous rules, it is possible to abandon consistency.
It simply tolerate incoherences between multiple replicas.
The output of an operation can be false while its state is synchronized with the other replicas.

% It is similar to the propagation of sound, light or gravity.
% When an explosion happens, not everybody hears it at the same time : there is inconsistency in the experience.


% TODO continue