\section{Javascript}

Javascript was released in a hurry, without a strong and directive philosophy.
During its evolution, it snowballed with different features to accommodate the community, and the usage it was made on the web. As a result Javascript contains various, and sometimes conflicting, programming paradigms.
It borrows its syntax from a procedural language, like C, and the object notation from an object-oriented language, like Java, but it provides a different inheritance mechanism, based on prototypes. Most of the implementation adopt an event-based paradigm, like the DOM\ftnt{http://www.w3.org/DOM/} and node.js\ftnt{https://nodejs.org/}.
And finally, event though it is not purely functional like Haskel, Javascript borrows some concepts from functional programming.

In this section, we focus on the last two programming paradigm, functional programming and event-based programming.
Javascript exposes two features from functional programming that are particularly adapted for event-based programming.
Namely, it treats functions as first-class citizen, and allows them to close on their defining context, to become closures.
% TODO In the next section, we explain this two features
% TODO and in the second section, we explain event-based programming and why these two features are good for event-based programming

\subsection{Functions as First-Class citizens}

\cit{All problems in computer science can be solved by another level of indirection}{Butler Lampson}

Javascript treats function as first-class citizens.
One can manipulate functions like any other type (number, string ...).
She can store functions in variables or object properties, pass functions as arguments to other functions, and write functions that return functions.

The most common usage examples of these features, are the methods \texttt{Map}, \texttt{Reduce} and \texttt{filter}.
In the example below, the method \texttt{map} expect a function to apply on all the element of an array to modify its content, and output a modified array.
A function expecting a function as a parameter is considered to be a higher-order function. \texttt{Map}, \texttt{Reduce} and \texttt{Filter} are higher-order functions.

\begin{code}
  [4, 8, 15, 16, 23, 42].map(function firstClassFunction(element) {
    return element + 1;
  });
  // -> [5, 9, 16, 17, 24, 43]
\end{code}

Higher-order functions provide a new level of indirection, allowing abstractions over functions.
To understand this new level of abstraction, let's briefly summarize the different abstractions on the execution flow offered by programming paradigms.
In imperative programming, the control structures allow to modify the control flow. That is, for example, to execute different instructions depending on the state of the program.
Procedural programming introduces procedures, or functions. That is the possibility to group instructions together to form functions.
They can be applied in different contexts, thus allowing a new abstraction over the execution flow.
% It encourages to abstract program states so that the same function can be applied in different places to apply its behavior.

So, higher-order functions add another level of abstraction.
It allows to dynamically modify the control of the execution flow.
The ability to manipulate functions like any other value allows to abstract over functions, and behavior.
% TODO continue this, there is a lot to say about HOF

Higher-order functions replace the needs for some Object oriented programming design patterns.\ftnt{http://stackoverflow.com/a/5797892/933670} Though object oriented programming doesn't exclude higher-order functions.

They are particularly interesting when the behavior of the program implies to react to inputs provided during the runtime, as we will see later.
Web servers, or graphical user interfaces, for examples, interact with external events of various types.

% TODO transition : higher-order functions makes use of closure to implement the lexical scope in mutable programming languages. (reformulate)

\subsection{Lexical Scoping}

Closures are indissociable from the concept of lexical environment.
To understand the former, it is important to understand the latter first.

\subsubsection{Lexical environment}

A variable is the very first level of indirection provided by programming languages and mathematics.
It is is a binding between a name and a value.
Mutable like in imperative programming to represent the reality of memory cells, or immutable like in mathematics and functional programming.
These bindings are created and modified during the execution.
They form a context in which the execution takes place.
To compartmentalize the execution, a context is also compartmentalized.
A certain context can be accessed only by a precise portion of code.
Most languages defines the scope of this context using code blocks as boundaries.
That is known as lexical scoping, or static scoping.
The variables declared inside a block represent the lexical environment of this block.
These lexical environments are organized following the textual hierarchy of code blocks.
The context available from a certain block of code, that is set of accessible variable, is formed as a cascade of the current lexical environment and all the parent lexical environment, up to the global lexical environment.

% TODO draw the schema for a lexical environment here

\subsubsection{Javascript lexical environment}
\ftnt{http://www.ecma-international.org/ecma-262/5.1/\#sec-10.2}

Javascript implement lexical scoping with function definitions as boundaries, instead of code blocks.
The code below show a simple example of lexical scoping in Javascript.

\begin{code}
  var a = 4;
  var c = 6;
  function f() {
    var b = 5;
    var c = 0;
    // a and b are accessible here.
    return a + b + c;
  }

  f(); // -> 9

  // b is not accessible here :
  a + b + c; // -> ReferenceError: b is not defined
\end{code}

Lexical scoping, or statical scoping, implies that the lexical environment are known statically, at compile time for example.
But Javascript is a dynamic language, it doesn't truly provide lexical scoping.
In Javascript, the lexical environments can be dynamically modified using two statements : \texttt{with} and \texttt{eval}.
We explain in details the Javascript lexical scope in section \ref{??? Compiler stuff}

\subsection{Closure}

\cit{An object is data with functions. A closure is a function with data.}{John D. Cook}

A closure is the association of a first-class function with its context.
When a function is passed as an argument to an higher-order function, she closes over its context to become a closure.
When a closure is called, it still has access to the context in which it was defined.
The code below show a simple example of a closure in Javascript.
The function \texttt{g} is defined inside the scope of \texttt{f}, so it has access to the variable \texttt{b}.
When \texttt{f} return \texttt{g} to be assigned in \texttt{h}, it becomes a closure.
The variable \texttt{h} holds a closure referencing the function \texttt{g}, as well as its context, containing the variable \texttt{b}.
The closure \texttt{h} has access to the variable \texttt{b} even outside the scope of the function \texttt{f}.

\begin{code}
  function f() {
    var b = 4;
    return function g(a) {
      return a + b;
    }
  }

  var h = f();
  // b is not accessible here :
  b; // -> ReferenceError: b is not defined

  // h is the function g with a closure over b :
  h(5) // -> 9
\end{code}


\subsection{Current and Future trends}

\paragraph{ES6 / 7}

\comment{TODO}

Code reuse.
Why it never worked ?




\comment{em-scripten}

https://github.com/kripken/emscripten
Javascript is a target language for LLVM, therefor everything can compile to Javascript : JS is the assembler of the web.

\comment{Isomorphic Javascript}

Server-side Javascript

https://www.meteor.com/
https://facebook.github.io/flux/
Javascript can be executed both on the client and the server.
That allow use-cases never possible before (server pre-rendering, same team ...)

\comment{Reactive}

http://facebook.github.io/react/
Javascript is used to model the flow of propagation of state in a web application




\endinput

\subsection{About closures and an alternative to the fluxional execution model}
% From a loose file, somewhere in one of my repositories

A closure is a function associated with an environment.
In Javascript it is declared like this :


\begin{code}
function closureCreator(env) {
  return function closure() {
    console.log('my environement is ' + env);
  }
}

closureCreator('dynamically generated')();
\end{code}

The same function can lead to multiple closures.

\begin{code}
closureCreator('privately enclosed')();
closureCreator('multiple time privately enclosed')();
\end{code}

In Javascript, a callback is very often a closure.
It allows to share a scope between the caller and the callee in an asynchronous operation, as it's not always possible to pass custom parameters to callback functions.
A callback is kind of a fluxion.
It is executed when it receives a message, and can send more messages (asynchronous operation)

But one function can lead to multiple closures, as seen above.
While a fluxion is a one-one assciation between a scope and a function.
Therefore, there must be one fluxion per closure.



The alternative version of closure we studied a few months ago proposes a one-n association between fluxion and context.
Therefore, allowing the same association as the function for closures.

A fluxion can have multiple contexts, identified by an identifier.
Depending on the message the fluxion receives, the context presented to the fluxion is not the same.
This has the advantage of presenting a *group by* mechanism.

% TODO continue with closure


% TODO transition, Higher-order functions and closures are very handy in turn-based programming.
% Turn based programming is the programming model of the event-loop, which is the concurrent model for I/O bound applications.


% TODO Promises ? where are you ?
% http://pouchdb.com/2015/05/18/we-have-a-problem-with-promises.html


---

Some facts to include :
https://www.destroyallsoftware.com/talks/the-birth-and-death-of-javascript
The Atom editor is written in Javascript node.js.
Now, major PaaS (which one) support node.js by default.
Heroku support Python, Java, Ruby, Node.js, PHP, Clojure and Scala
Amazon Lambda Web service support node.js in priority.
npm raises 8m.
http://techcrunch.com/2015/04/14/popular-javascript-package-manager-npm-raises-8m-launches-private-modules/

% >>> I want to say that Javascript is now broadly used.
% Let's just look at the numbers : Javascript is the most popular language on Github, and npm has more package than any other package manager.
% Javascript has the more broadly deployed runtime.
% ... and so on
% >>> the conclusion is : Javascript is now a major language, and it is more than worth the consideration we are giving it in this PhD thesis.

With numbers from Worldline about Java and Javascript usages.

It ends with some success stories with Javascript (Paypal, Linkedin and so on ...).

-> There is a growing mass of developers for Javascript.
And even for those who don't program in Javascript, there is transpilers.
Javascript is here to stay : http://www.javaworld.com/article/2077224/learn-java/is-javascript-here-to-stay-.html (an article from 1996, where Java was still the hot new language)
