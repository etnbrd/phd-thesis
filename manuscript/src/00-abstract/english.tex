
\begin{abstract}

Most of the now popular web services started as small projects created by few individuals, and grew exponentially.
Internet supports this growth because it extends the reach of our communications world wide, while reducing their latency.
During its development, an application must grow exponentially, otherwise the risk is to be outpaced by the competition.

In the beginning, it is important to verify quickly that the service can respond to the user needs: \textit{Fail fast}.
Languages like Ruby or Java became popular because they propose a productive approach to iterate quickly on user feedbacks.
A web application that correctly responds to user needs can become viral.
Eventually, the application needs to be efficient to cope with the traffic increase.

But it is difficult for an application to be at once productive and efficient.
When the user base become too important, it is often required to switch the development approach from productivity to efficiency.
No platform conciliates these two objectives, so it implies to rewrite the application into an efficient execution model, such as a pipeline.
It is a risk as it is a huge and uncertain amount of work.
To avoid this risk, this thesis proposes to maintain the productive representation of an application with the efficient one.

Javascript is a productive language with a significant community.
It is the execution engine the most deployed, as it is present in every browser, and on some servers as well with \textit{Node.js}.
It is now considered as the main language of the web, ousting Ruby or Java.
Moreover, the Javascript event-loop is similar to a pipeline.
Both execution models process a stream of requests by chaining independent functions.
Though, the event-loop supports the needs in development productivity with its global memory, while the pipeline representation allows an efficient executions by allowing parallelization.

This thesis studies the possibility for an equivalence to transform an implementation from one representation to the other.
With this equivalence, the development team can follow the two approaches concurrently.
It can continuously iterate the development to take advantage of their conflicting objectives.

This thesis presents a compiler that allows to identify the pipeline from a Javascript application, and isolate its stages into fluxions.
A fluxion is named after the contraction between function and flux.
It executes a function for each datum on a stream.
Fluxions are independent, and can be moved from one machine to the other, so as to cope with the increasing traffic.
The development team can begin with the  productivity of the event-loop representation.
And with the transformation, it can progressively iterate to reach the efficiency of the pipeline representation.

% By transforming a Javascript application into a fluxional pipeline, 
% And it probably can make it efficient with significantly less effort than the current rewrite required.

% This work was done in collaboration between the company Atos Worldline, and the DICE team from CITI laboratory, INSA de Lyon.
% The goal for Worldline was to develop the state of the art in the field of web application development and hosting.
% The goal for the DICE team was to conduct a research activity within an industrial context to be in contact with its economic constraints.


\end{abstract}