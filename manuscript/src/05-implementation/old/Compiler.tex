% # Explanation of the concept

% ## Turn-based programming.

% (see presentation on Dues)
% -> single-thread, no wait, no block and so on
% Shared heap -> no mutex, no synchronization, so it is good scalability


% Turn-based programming is an event-loop.
% It is the execution of queued events one after the other.
% An event is the association of a callback and a message.
% The callback is a small Javascript Program, designed to process the message.
% During its turn, the callback executes, and can queue events : that is register callback to be executed during a next turn.
% TODO what I mean exactly by queue events ? -> the distinction between the asynchronous operation, and the resulting event.

% ## Pipeline

% So a callback sends messages to other callbacks.
% -> It is exactly like a pipeline.
% However, all the callbacks share the same heap.
% So it is not possible to distribute the different callbacks without synchronization of this heap, or splitting the heap for each callback.
% TODO state VERY clearly this problem, it is at the core of my thesis.

% So, how to split the heap so that each callback has its own exclusive heap ?

\comment{From here, the reader should be confortable with the event-loop, and the analogy we drawn between the event-loop and a pipeline.
The problematic is now clear : how to split the heap so that each asynchronous callback has its own exclusive heap ?}

\section{Callback identification}

\subsection{\comment{TODO}}

\section{Callback isolation}

We explain in this section the compilation process we developped to isolate the memory access for each callbacks.
The result of this process should be two-fold. First each callback should have an exclusive access on a region of the memory. So that two different callback can be executed in parallel. And it should be clear for each callback, what are the variable needed from upstream callbacks, and what are the variable to send downstream.

\subsection{Propagation of variables}


\subsubsection{Scope identification}

In section \ref{??? Javascript scope / closure}, we explained that Javascript is roughly lexically scoped.
A consequence is that the declaration of contexts can be inferred statically.
For example, in a lexically scoped, strongly typed, compiled language, the compiler know the content of each scope during compile time, and can prepare the memory stack to store the variables in each scope.

In most languages, the memory is in two parts : the stack, and the heap.
The stack is statically scoped, and its layout is known at compile time.
The heap, on the other hand is dynamically allocated. Its layout is built at run time.

But Javascript is a dynamic language, perhaps the most dynamic of all languages.
It doesn't have this distinction between stack and heap. Every variable is dynamically allocated on the heap.
That induce two consequences.
The first is that Javascript provides two statements to dynamically modify the lexial scope : \texttt{eval} and \texttt{with}.
The second is that to know the layout of the heap, we need to use static analysis tools.
In the next two sections, we adress these two consequences.

\subsubsection{Break the lexical scope} \label{???:breakscope}

Without these statements, \texttt{eval} and \texttt{with}, Javascript is lexically scoped. It is possible to infer the scope of each variable at compile time.


The \texttt{with} statement continue the execution using an expression as the lexical scope.
As the provided expression is dynamically evaluated, it is possible to dynamically modify the lexical scope.
The code snippet below show an example of such a situation.

\begin{code}
var aliveCat = {isAlive: true};
var deadCat = {isDead: true}

with (Math.random() > 0.5 ? aliveCat : deadCat) {
  isAlive;
  // Half the time -> ReferenceError: isAlive is not defined
  // Half the time -> true;
}
\end{code}

The variable \texttt{isAlive} is defined only in the object \texttt{aliveCat}.
The presence of the variable \texttt{isAlive} in the lexical environment within the \texttt{with} statement cannot be determined statically, as the lexical environment is dynamically linked to either \texttt{aliveCat} or \texttt{deadCat}.

Note that the MDN reference page on \texttt{with}\ftnt{https://developer.mozilla.org/en-US/docs/Web/JavaScript/Reference/Statements/with} says that \textit{Using \texttt{with} is not recommended, and is forbidden in ECMAScript 5 strict mode.}

The \texttt{}

% TODO and specify that Javascript is roughly lexically scoped : it is not completely lexically scoped, and five examples to backup that.

Not to be mistaken with the \texttt{this} operator.
It is possible to dynamically change the content of an object,  
% TODO continue this paragraph about how the this operator change the properties of an object. 
% Does it change the lexical scope, if that object is actually used as a context elsewhere ? -> No, I don't think so.
% But ask on SO, just to be sure.

\begin{code}
function stuff() {
  this.x = 42;
}

stuff.call({})

\end{code}

% Javascript is lexically scoped, therefore we can identify the scope of variable statically.
% (At the exception of eval and with : with is forbidden from strict mode, so that is not a bigdeal, howether, eval is sometimes used in smart ways, but most of the time it is monomorphic (I don't exactly know what that means, I heard from Floreat, it must be something related to PL community)).

% The compiler identifies the variables shared by multiple callbacks from their scope.
% TODO explain this in depth.
% Function scope, closures, and so on ...



However, even if Javascript is lexically scoped, the memory is still dynamicall allocated and manipualeted, so that it is not possible to actually infer the memory layout at compiler time only with lexical scope analysis, and without deeper static analysis.

\subsubsection{Scope Leaking}

% Javascript uses a pass-by-sharing paradigm.
% That means that sometimes the argument of a call are passed by value, sometimes by reference (atomic data type (number, string, bool) -> by value, complex data type (objects) -> by reference).
% That means that the modification of a local variable can affect variable in seemingly unrelated scopes.
% It seems that the points-to analysis is what is used to find stuffs like that (side-effects ?).

% TODO what we are talking about here are aliases.

% TODO I am stating here that in low-level language, the memory access is so fine, that it is difficult to exactly pin down the memory layout in term of object, it is rather seen as a big array of memory adresses.
% While in higher-level language, like Javascript, the memory access is at the property level (it is not possible to access memory down to the adress), so it could be easier (maybe, just not harder) to infer the dynamic memory layout from source.
To infer the layout of the heap at compile time, static analysis tools are used, like the points-to analysis, developped by Andersen in its PhD thesis \cite{Andersen1994}.
For such analysis, the memory is splitted at the access scale.
In low-level languages, like C/C++, the memory is mainly managed by the developer. Allowing access to the memory at a small grained scale : up to the address.
It impose the analysis to split the memory to the adress scale in some cases.
% TODO Backup that, HEAVILY
In higher-level languages, like Javascript, the developer cannot access the memory to the adress scale.
The memory is accessed at a coarser scale : the property scale.
(At the exception of some arrays and buffers, that mimic, and are mapped to actual memory adresses for performance reasons.)
% TODO find exactly the references for these buffers : I think of ArrayBuffer, and sharedArray ... but I am not sure. Need more inspection.

\subsubsection{Propagation of execution and variables}

For the execution of each callback / stage, the corresponding part of the state is local, and the rest is remote, and inaccessible.
We are going to explain why it must remain inaccessible.

While a callback is executing a request, the previous callback (the up stream callback) is executing the next request.
The next request will arrive at the current callback some time in the future.
The modification done in the state of the upstream callback will propagate only later in the current callback.
The state of the upstream callback is in a different time frame than the state of the current callback.

To really understand that, we need to compare this execution with the execution on a unique event-loop.
If the current callback executes, then the upstream callback might have, or might not have started to execute the next request.
But as soon as the current callback executes, the modifications done on the states, are immediatly propagated, so that the upstream callback can take them into account for the next request.

However, if the two callbacks are distant, then the modification of the current callback will not immediatly propagate to the upstream callback.
During the propagation, the upstream callback might execute requests than would not be aware of the state modification from the current callback (from downstream).
That is why we say the upstream callback and the current callback are in two different time frame.
Propagating the state modification upstream is like going backward in time, it is impossible.
That is why the execution, and the state modification propagation must always flow downstream.

As a note, I must add that if an upstream and a downstream callbacks are on the same event-loop, then this doesn't apply. it is like a loop in the time : the modification immediatly propagate from downstream to upstream.






% The execution progress downstream, following the message stream.
% TODO state very clearly this proposition, it is the second core of my thesis (and I love the idea, it relates directly to reality, graivity, and the fabric of the universe <3).

% Because the propagation of the modification is not instantaneous, going back upstream is like going backward in time : it is impossible.
% Therefore, a variable cannot be read upstream a write.
% And it cannot be write downstream either.

% In other words, only one callback can write on a variable -> seems obvious from previous sections.


% In promises, because the heap is not shared, things are less restrictive.
% Multiple stages can read and write the same variable, because the propagation of modification is instantaneous, due to the shared heap.





% TODO write about what it implies to detect continuation in variable, or other expressions.

% Why can we only detect continuations declared in situ.
% If a continuation is passed as a variable, we don't know for sure what is the function associated with, and the closure of that function.
