\subsection{Real test case} \label{chapter5:flx:evaluation}

This section presents a test of the compiler on a real application, gifsockets-server\ftnt{https://github.com/twolfson/gifsockets-server}.
This test proves the possibility for an application to be compiled into a network of independent parts.
It shows the current limitations of this isolation and the modifications needed on the application to circumvent them.
This section then presents future works.

\begin{code}[js, caption={Simplified version of gifsockets-server},label={lst:gifsocket}]
var express = require('express'),
    app = express(),
    routes = require('gifsockets-middleware'), //@\label{lst:gifsocket:gif-mw}@
    getRawBody = require('raw-body');

function bodyParser(limit) { //@\label{lst:gifsocket:bodyParser}@
  return function saveBody(req, res, next) { //@\label{lst:gifsocket:saveBody}@
    getRawBody(req, { //@\label{lst:gifsocket:getRawBody}@
      expected: req.headers['content-length'],
      limit: limit
    }, function (err, buffer) { //@\label{lst:gifsocket:callback}@
      req.body = buffer;
      next(); //@\label{lst:gifsocket:next}@
    });
  };
}

app.post('/image/text', bodyParser(1 * 1024 * 1024), routes.writeTextToImages); //@\label{lst:gifsocket:app.post}@
app.listen(8000);
\end{code}

This application, simplified in listing \ref{lst:gifsocket}, is a real-time chat using gif-based communication channels.
It was selected in a previous work \cite{Brodu2015} from the \texttt{npm} registry because it depends on \texttt{express}, it is tested, working, and simple enough to illustrate this evaluation.
The server transforms the received text into a gif frame, and pushes it back to a never-ending gif to be displayed on the client.

On line \ref{lst:gifsocket:app.post}, the application registers two functions to process the requests received on the url \texttt{/image/text}.
The closure \texttt{saveBody}, line \ref{lst:gifsocket:saveBody}, returned by \texttt{bodyParser}, line \ref{lst:gifsocket:bodyParser}, and the method \texttt{routes.write\-Text\-To\-Images} from the external module \texttt{gifsockets-middleware}, line \ref{lst:gifsocket:gif-mw}.
The closure \texttt{saveBody} calls the asynchronous function \texttt{getRawBody} to get the request body.
Its callback handles the errors, and calls \texttt{next} to continue processing the request with the next function, \texttt{routes.write\-Text\-To\-Images}.

\subsubsection{Compilation}

We compile this application with the compiler detailed in section \ref{section:compiler}.
Listing \ref{lst:flx-gifsocket} presents the compilation result.
The function call \texttt{app.post}, line \ref{lst:gifsocket:app.post}, is a rupture point.
However, its callbacks, \texttt{bodyParser} and \texttt{routes.write\-Text\-To\-Images} are evaluated as functions only at runtime.
For this reason, the compiler ignores this rupture point, to avoid interfering with the evaluation.

\begin{code}[flx, caption={Compilation result of gifsockets-server},label={lst:flx-gifsocket}]
flx main & express {req}
>> anonymous_1000 [req, next]
  var express = require('express'),
      app = express(),
      routes = require('gifsockets-middleware'), //@\label{lst:flx-gifsocket:gif-mw}@
      getRawBody = require('raw-body');

  function bodyParser(limit) { //@\label{lst:flx-gifsocket:bodyParser}@
    return function saveBody(req, res, next) { //@\label{lst:flx-gifsocket:saveBody}@
      getRawBody(req, { //@\label{lst:flx-gifsocket:getRawBody}@
        expected: req.headers['content-length'], //@\label{lst:flx-gifsocket:req.headers}@
        limit: limit
      }, >> anonymous_1000);
    };
  }

  app.post('/image/text', bodyParser(1 * 1024 * 1024), routes.writeTextToImages); //@\label{lst:flx-gifsocket:app.post}@
  app.listen(8000);

flx anonymous_1000
-> null
  function (err, buffer) { //@\label{lst:flx-gifsocket:callback}@
    req.body = buffer; //@\label{lst:flx-gifsocket:buffer}@
    next(); //@\label{lst:flx-gifsocket:next}@
  }
\end{code}

The compiler detects a rupture point : the function \texttt{get\-Raw\-Body} and its anonymous callback, line \ref{lst:gifsocket:callback}.
It encapsulates this callback in a fluxion named \texttt{anonymous\_\-1000}.
The callback is replaced with a stream placeholder to send the message stream to this downstream fluxion.
The variables \texttt{req} and \texttt{next} are appended to this message stream, to propagate their value from the \texttt{main} fluxion to the \texttt{anonymous\_\-1000} fluxion.

When \texttt{anonymous\_\-1000} is not isolated from the \texttt{main} fluxion, as if they belong to the same group, the compilation result works as expected.
The variables used in the fluxion, \texttt{req} and \texttt{next}, are still shared between the two fluxions.
Our goal is to isolate the two fluxions, to be able to safely parallelize their executions.

\subsubsection{Isolation}

In listing \ref{lst:flx-gifsocket}, the fluxion \texttt{anonymous\_1000} modifies the object \texttt{req}, line \ref{lst:flx-gifsocket:buffer}, to store the text of the received request, and it calls \texttt{next} to continue the execution, line \ref{lst:flx-gifsocket:next}.
These operations produce side-effects that should propagate in the whole application, but the isolation prevents this propagation.
Isolating the fluxion \texttt{anonymous\_1000} produces runtime exceptions.
We detail in the next paragraph, how we handle this situation to allow the application to be parallelized.

\paragraph{Variable \texttt{req}}

The variable \texttt{req} is read in fluxion \texttt{main}, lines \ref{lst:flx-gifsocket:getRawBody} and \ref{lst:flx-gifsocket:req.headers}.
Then its property \texttt{body} is associated to \texttt{buffer} in fluxion \texttt{anonymous\_1000}, line \ref{lst:flx-gifsocket:buffer}.
The compiler is unable to identify further usages of this variable.
However, the side effect resulting from this association impacts a variable in the scope of the next callback, \texttt{routes.writeTextToImages}.
We modified the application to explicitly propagate this side-effect to the next callback through the function \texttt{next}.
We explain further modification of this function in the next paragraph.

\paragraph{Closure \texttt{next}}

The function \texttt{next} is a closure provided by the \texttt{express} \texttt{Router} to continue the execution with the next function to handle the client request.
Because it indirectly relies on the variable \texttt{req}, it is impossible to isolate its execution with the \texttt{anonymous\_\-1000} fluxion.
Instead, we modify \texttt{express}, so as to be compatible with the fluxionnal execution model.
We explain the modifications below.

\begin{code}[flx, caption={Simplified modification on the compiled result},label={lst:mflx-gifsocket}]
flx anonymous_1000
-> express_dispatcher
  function (err, buffer) { //@\label{lst:mflx-gifsocket:callback}@
    req.body = buffer; //@\label{lst:mflx-gifsocket:buffer}@
    next_placeholder(req, -> express_dispatcher); //@\label{lst:mflx-gifsocket:next-placeholder}@
  }

flx express_dispatcher & express {req} //@\label{lst:mflx-gifsocket:express-dispatcher}@
-> null
  function (modified_req) {
    merge(req, modified_req);
    next(); //@\label{lst:mflx-gifsocket:next}@
  }
\end{code}

In listing \ref{lst:gifsocket}, the function \texttt{next} is a continuation allowing the anonymous callback, line \ref{lst:gifsocket:callback}, to call the next function to handle the request.
To isolate the anonymous callback into \texttt{anonymous\_\-1000}, \texttt{next} is replaced by a rupture point.
This replacement is illustrated in listing \ref{lst:mflx-gifsocket}.
The \texttt{express} \texttt{Router} registers a fluxion named \texttt{express\_\-dispatcher}, line \ref{lst:mflx-gifsocket:express-dispatcher}, to continue the execution after the fluxion \texttt{anonymous\_\-1000}.
This fluxion is in the same group \texttt{express} as the \texttt{main} fluxion, hence it has access to the original variable \texttt{req}, and to the original function \texttt{next}.
The call to the original \texttt{next} function is replaced by a placeholder to push the stream to the fluxion \texttt{express\_\-dispatcher}, line \ref{lst:mflx-gifsocket:next-placeholder}.
The fluxion \texttt{express\_\-dispatcher} receives the stream from the upstream fluxion \texttt{anonymous\_\-1000}, merges back the modification in the variable \texttt{req} to propagate the side effects, and finally calls the original function \texttt{next} to continue the execution, line \ref{lst:mflx-gifsocket:next}.

After the modifications detailed above, the server works as expected.
The isolated fluxion correctly receives, and returns its serialized messages.
The client successfully receives a gif frame containing the text.