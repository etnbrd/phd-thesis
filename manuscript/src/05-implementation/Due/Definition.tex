% \subsection{Definition} \label{chapter5:due:definition}

\subsection{Definition} \label{chapter5:due:definition}

A Due is an object used as placeholder for the eventual outcome of a deferred operation.
% Dues are a simplification of the Promise specification.
They are essentially similar to ECMAScript Promises\ftnt{http://www.ecma-international.org/ecma-262/6.0/\#sec-promise-objects}, except for the convention to hand back outcomes.
% They leave the control over the conditional execution over the outcome to the developer.
They use the \textit{error-first} convention, like \textit{Node.js}, as illustrated line \ref{lst:due:error-first} in listing \ref{lst:due}.
The implementation of Dues and its tests are available online\ftnt{https://www.npmjs.com/package/due}.
% A more in-depth description of Dues and their creation follows in the next paragraphs.
% The \texttt{mock} method is implemented in listing \ref{lst:due-creation}.
% While a promise expects two continuations, \texttt{onSuccess} and \texttt{onErrors}, the method \texttt{then} of a due expects only one continuation, following the convention \textit{error-first}.
% \footnotemark{\ref{ftn:error-conventions}}
% \footnotemark{\ref{ftn:error-first}}.

\subsubsection{Usage}

\begin{code}[js, %
             caption={Example of a due}, %
             label={lst:due}] %
var my_fn_due = require('due').mock(my_fn); //@\label{lst:due:mock}@

var due = my_fn_due(input);

due.then(function continuation(error, result) { //@\label{lst:due:error-first}@
  if (!error) {
    console.log(result);
  } else {
    throw error;
  }
});
\end{code}


In listing \ref{lst:due}, the function \texttt{my\_fn\_due} synchronously returns a due as a placeholder for its outcome.
The \texttt{then} method of the due allows to define a continuation to continue the execution after retrieving the outcome, like line \ref{lst:due:error-first}.
If the deferred operation is synchronous, the Due settles during its creation and the \texttt{then} method immediately calls this continuation.
If the deferred operation is asynchronous, this continuation is called during the Due settlement.

\subsubsection{Creation} \label{chapter5:due:definition:creation}

\begin{code}[js, %
             caption={Creation of a due}, %
             label={lst:due-creation}] %
Due.mock = function(my_fn) { //@\label{lst:due-creation:mock}@
  return function mocked_fn() { //@\label{lst:due-creation:mocked}@
    var _args = Array.prototype.slice.call(arguments),
        _this = this;

    return new Due(function(settle) {  //@\label{lst:due-creation:new}@
      _args.push(settle);  //@\label{lst:due-creation:push}@
      my_fn.apply(_this, _args); //@\label{lst:due-creation:call}@
    })
  }
}
\end{code}


% A due is often created inside the function which returns it, like in listing \ref{lst:due-creation}.
In listing \ref{lst:due}, line \ref{lst:due:mock}, the \texttt{mock} method wraps the original function \texttt{my\_fn} in a Due-compatible function \texttt{mocked\_fn}.
The \texttt{mock} method is detailed in listing \ref{lst:due-creation} to illustrate the creation of a Due.
% The rest of this code is similar to the Promise example, listing \ref{lst:then}.
It returns a Due compatible function, \texttt{mocked\_fn}, line \ref{lst:due-creation:mocked}.
That is a function that returns a Due, instead of expecting a continuation.

At the execution of \texttt{mocked\_fn} the Due to be returned is created line \ref{lst:due-creation:new}, with the original function passed as argument.
The original function \texttt{my\_fn} is executed during the creation of the Due.
The \texttt{settle} function provided is passed as a continuation line \ref{lst:due-creation:push} for the original function to settle the returned Due. %, synchronously or asynchronously.
% Therefore, the \texttt{settle} function is pushed at the end of the list of arguments, line \ref{lst:due-creation:push}.
% Indeed, the operation might be synchronous, or asynchronous.
% The callback invokes the deferred operation with this list of arguments, and the current context, line \ref{lst:due-creation:call}.
% \texttt{my\_fn} being asynchronous, it expects a callback as last argument : \texttt{settle}.
When the original function completes, it calls \texttt{settle} to settle the Due and save the outcome.
This outcome can then be retrieved with the continuation provided by the \texttt{then} method.
% A Due is in one of two mutually exclusive states: settled or pending.

% This continuation is defined by the \texttt{then} method.
% After the settlement of the Due, its continuation is executed with the outcome.
% Dues expose a \texttt{then} method expecting a continuation to continue the execution after its settlement.


\subsubsection{Composition}

Dues arrange the execution flow as a chain of actions to carry on inputs.
The composition of Dues in a chain is illustrated in listing~\ref{lst:dues-sequence}.
It is similar to the composition of Promises explained in the previous chapter, section~\ref{chapter4:event-loop:promise}, page~\pageref{chapter4:event-loop:promise}.
% explained in section \ref{chapter4:event-loop:promise}.
% It is the same than for Promises,
% This composition is explained in details in section \ref{section:definitions:promise}. %, and illustrated specifically for Dues in listing \ref{lst:dues-sequence}.
% Through this chained composition,

\begin{code}[js, %
             caption={Dues are chained like Promises}, %
             label={lst:dues-sequence}] %
var Due = require('due');

var my_fn_due_1 = Due.mock(my_fn_1),
    my_fn_due_2 = Due.mock(my_fn_2),
    my_fn_due_3 = Due.mock(my_fn_3);

my_fn_due_1(input)
.then(my_fn_due_2) //@\label{lst:due-sequence:then}@
.then(my_fn_due_3)
.then(console.log);
\end{code}

The \texttt{then} method of the current Due returns an intermediary Due that settles when the Due returned by the passed continuation settles.
For example, in listing \ref{lst:dues-sequence} the Due returned by the \texttt{then} method line \ref{lst:due-sequence:then} settles when the Due returned by its continuation \texttt{my\_fn\_due\_2} settles.
It allows to chain continuations one after the other like a pipeline, instead of the nested composition of continuations.

% This simplified specification adopts the same convention than \textit{Node.js} for continuations to hand back outcomes.


% Therefore, the equivalence between a continuation and a Due is trivial.



% This equivalence, and its composition are explained in details in section \ref{section:equivalence}.
% Dues are admittedly tailored for this work, hence, they are not designed to be written by developers, like Promises are.

% The next section presents the equivalence between continuations and Dues.


% In listing \ref{lst:due}, \texttt{due} is settled when the function \texttt{settle} is called.
% If \texttt{due} is settled, a call to \texttt{due.then(onSettlement)} immediately call the function \texttt{onSettlement}.
% A due is pending if it is not settled.
% A due is resolved if it is settled or if it has been linked with another due.
% Attempting to settle a resolved due has no effect.
% A resolved due may be pending or settled, while an unresolved due is always in the pending state.
% The \texttt{Due} object only exposes the \texttt{then} method.
% \textbf{\texttt{Due.prototype.then(onSettlement)}}\\
% Appends settlement handlers to the due, and returns a new due resolving to the return value of the called handler.
% If the value is a \textit{thenable}, \textit{i.e.} has a method \texttt{then}, the returned due will follow that \textit{thenable}, adopting its eventual state; otherwise the returned due will be fulfilled with the value.
% We present in appendix \ref{section:dueimpl} a simple implementation of Due in Javascript.
