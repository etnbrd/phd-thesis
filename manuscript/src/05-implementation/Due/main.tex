\section{First Step : Due Compiler} \label{chapter5:due}

% The previous chapter presented globally the state of the art in designing systems to scale in performance, and in maintenance.
% It refined the scope of this thesis to the study of the opposition between maintenance scalability and performance scalability in streaming web applications.
% It concluded with the objectives of this thesis, which is to find an equivalence between the two opposed organizations.
% The maintenance scalability organization, supported by modular programming, higher-order programming and a global memory store.
% The performance scalability organization, supported by the parallelism of memory and execution distribution.
% This section presents the first step in the transformation from the event-driven execution model to the pipeline architecture, as presented in figure \ref{fig:roadmap}.
% That is to identify and extract a pipeline of execution inside an application following the event-driven execution model.
% In this work, we focus on Javascript, and specifically on \textit{Node.js} applications.
%
% Javascript allows higher-order programming.
% It allows to manipulate functions like any object.
% For example to link them to react to asynchronous events, \textit{e.g.} user inputs and remote requests.
% These asynchronously triggered functions are named callbacks, and allow to efficiently cope with the distributed and inherently asynchronous architecture of the Internet.
% To execute a suite of asynchronous functions, callbacks are nested one into the other.
% This nesting, if not organized properly, can result in unreadable layer of callbacks, commonly presented as \textit{callback hell}\ftnt{http://maxogden.github.io/callback-hell/}, or \textit{pyramid of doom}.
%
% Promises are another way to organize a suite of asynchronous operations avoiding this callback hell.
% They organize the operations as a well-defined pipeline.
% Moreover, Promises provide additional control over the asynchronous execution flow, than callbacks.
% They are part of the next version of the Javascript language, ECMAScript 6\ftnt{http://people.mozilla.org/~jorendorff/es6-draft.html}.
% To avoid the equivalence being unnecessarily incomplete, we present an alternative to Promise, called Due.
% Due organize the operations like Promises, as a well-defined pipeline, while discarding the unnecessary additional control over the asynchronous flow.
%

% It focuses on the identification of the chains of causality in continuations.
% Promises bring more control over the asynchronous flow than the chaining of causal sequentiality.
% But they force another control over the execution flow.
% According to the outcome of the operation, they call one function to continue the execution with the result, or another to handle errors.
% This conditional execution is indivisible from the Promise structure.
% As a result, Promises impose a convention on how to hand back the outcome of the deferred computation, while classic continuations leave this conditional execution to the developer.
% To rule out this differences between continuations and Promises, section \ref{chapter5:due} introduces a simpler specification to Promise, called Due.

% This section presents a compiler to identify the pipeline of operating underlying in a Javascript application. % using callbacks, and extract it to express it as Dues.
% The compiler expresses this pipeline as chains of Dues.

% Section \ref{chapter5:due:equivalence} explains the transformation from imbrications of continuations to sequences of Dues.
% Section \ref{chapter5:due:compiler} presents a compiler to automate the application of this equivalence.
% And finally, the developed compiler is evaluated in section \ref{chapter5:due:evaluation}.


% \subsection{From continuations to Promises} \label{seciton:definitions:analysis}

% As detailed in the previous sections, continuations provide the control over the sequentiality of the asynchronous execution flow.
% Promises improve this control to allow chained compositions, and unify the syntax for the synchronous and asynchronous paradigm.
% This chained composition brings a greater clarity and expressiveness to source codes.
% At the light of these insights, it makes sense for a developer to switch from continuations to Promises.
% However, the refactoring of existing code bases might be an operation impossible to carry manually within reasonable time.
% We want to automatically transform an imbrication of continuations into a chained composition of Promises.

% We identify two steps in this transformation.
% The first is to provide an equivalence between a continuation and a Promise.
% The second is the composition of this equivalence.
% Both steps are required to transform imbrications of continuations into chains of Promises.
% to be able to compose this equivalence for imbrications of continuations to obtain chains of Promises.

% Because Promises bring chained composition, the first step might seem trivial as it does not imply any imbrication to transform into chain.
% However, as explained in section \ref{section:definitions:promise}, Promises impose a control over the execution flow that continuations leave free.
% This control induces a common convention to hand back the outcome to the continuation.

% In the Javascript landscape, there is no dominant convention for handing back outcomes to continuations.
% In the browser, many conventions coexist.
% For example, \textit{jQuery}'s \texttt{ajax}\ftnt{http://api.jquery.com/jquery.ajax/} method expects an object with different continuations for success, errors and various other events during the asynchronous operation.
% \textit{Q}\ftnt{http://documentup.com/kriskowal/q/}, a popular library to control the asynchronous flow, exposes two methods to define continuations: \texttt{then} for successes, and \texttt{catch} for errors.
% % The conventions for continuations are very heterogeneous in the browser.
% On the other hand, the \textit{Node.js} API always used the \textit{error-first} convention, encouraging developers to provide libraries using the same convention.
% In this large ecosystem the \textit{error-first} convention is predominant.
% All these examples use different conventions than the Promise specification detailed in section \ref{section:definitions:promise}.
% They present strong semantic differences, despite small syntactic differences.
% % Some conventions include the conditional execution over the outcome, while other conventions let developers provide it.
% % These conventions uses different control-flow.

% To translate these different conventions into the Promises one, the compiler would need to identify them.
% Such an identification might be possible with static analysis methods such as the points-to analysis~\cite{Wei2014}, or a program logic~\cite{Gardner2013,Bodin2014}.
% However, it seems impracticable because of the number and semantical heterogeneity of these conventions.
% Indeed, in the browser, each library seems to provide its own convention.

% In this paper, we are interested in the transformation from imbrications to chains, not from one convention to another.
% The \textit{error-first} convention, used in \textit{Node.js}, is likely to represent a large, coherent code base to test the equivalence.
% Indeed contains currently more than 125 000 packages.
% For this reason, we focus only on the \textit{error-first} convention.
% Thus, our compiler is only able to compile code that follows this convention.
% The convention used by Promises is incompatible.
% We propose an alternative specification to Promise following the \textit{error-first} convention.
% In the next section we present this specification called Due.

% The choice to focus on \textit{Node.js} is also motivated by our intention to compare later the chained sequentiality of Promises with the data-flow paradigm.
% \textit{Node.js} allows to manipulate streams of messages.
% This proved to be efficient for real-time web applications manipulating streams of user requests.
% Both Promises and data-flow arrange the computation in chains of independent operations.
% % In section \ref{section:equivalence}, we explain the two steps of the transformation from continuations to Dues.

% Dues are further defined in section \ref{section:definitions}.

% This made Javascript a language of choice to develop both client and, more recently, server applications for the web.

% Callbacks are well-suited for small interactive scripts.
% But in a complete application, they are ill-suited to control the larger asynchronous execution flow.
% Their use leads to intricate imbrications of function calls and callbacks, commonly presented as \textit{callback hell}\ftnt{http://maxogden.github.io/callback-hell/}, or \textit{pyramid of doom}.
% This is widely recognized as a bad practice and reflects the unsuitability of callbacks in complete applications.
% Eventually, developers enhanced callbacks to meet their needs with the concept of Promise~\cite{Liskov1988}.

% Promises bring a different way to control the asynchronous execution flow, better suited for large applications.
% They fulfill this task well enough to be part of the next version of the Javascript language, ECMAScript 6\ftnt{http://people.mozilla.org/~jorendorff/es6-draft.html}.
% However, because Javascript started as a scripting language, beginners are often not introduced to Promises early enough.
% Most APIs use the classical callback approach encouraging beginner in this practice.
% Moreover, despite its benefits, the concept of Promise is not yet widely acknowledged.
% Developers may implement their own library for asynchronous flow control before discovering existing ones.
% There is such a disparity between the needs for and the adoption of Promises libraries, that there are almost 40 different implementations\ftnt{https://github.com/promises-aplus/promises-spec/blob/master/implementations.md}.

% With the upcoming introduction of Promise as a language feature, we expect an increase of interest, and believe that many developers will shift to this better practice.
% In this paper, we propose a compiler to automate this shift in existing code bases.
% We present the transformation from an imbrication of callbacks to a sequence of Promise operations, while preserving the semantic.

% Promises bring a better way to control the asynchronous execution flow, but they also impose a conditional control over the result of the execution.
% Callbacks, on the other hand, leave this conditional control to the developer.
% This paper focuses on the transformation from imbrication of callbacks to chain of Promises.
% To avoid unnecessary modifications on this conditional control, we introduce an alternative to Promises leaving this conditional control to the developer, like callbacks.
% We call this simpler specification Dues.
% Our approach enables us to compile legacy Javascript code and brings a first automated step toward full Promises integration.
% This simple and pragmatic compiler has been tested over 64 \textit{Node.js} packages from the node package manager (npm\ftnt{https://www.npmjs.com/}), 9 of them with success.

% In section \ref{section:definitions} we define callbacks, Promises and Dues.
% In section \ref{section:equivalence}, we explain the transformation from imbrications of callbacks to sequences of Dues.
% In section \ref{section:compiler}, we present a compiler to automate the application of this equivalence.
% In section \ref{section:evaluation}, we evaluate the developed compiler.

% \subsection{Definition} \label{chapter5:due:definition}

\subsection{Dues} \label{chapter5:due:definition}

A Due is an object used as placeholder for the eventual outcome of a deferred operation.
% Dues are a simplification of the Promise specification.
They are essentially similar to ECMAScript Promises\ftnt{http://www.ecma-international.org/ecma-262/6.0/\#sec-promise-objects}, except for the convention to hand back outcomes.
% They leave the control over the conditional execution over the outcome to the developer.
They use the \textit{error-first} convention, like \textit{Node.js}, as illustrated line \ref{lst:due:error-first} in listing \ref{lst:due}.
The implementation of Dues and its tests are available online\ftnt{https://www.npmjs.com/package/due}.
% A more in-depth description of Dues and their creation follows in the next paragraphs.
% The \texttt{mock} method is implemented in listing \ref{lst:due-creation}.
% While a promise expects two continuations, \texttt{onSuccess} and \texttt{onErrors}, the method \texttt{then} of a due expects only one continuation, following the convention \textit{error-first}.
% \footnotemark{\ref{ftn:error-conventions}}
% \footnotemark{\ref{ftn:error-first}}.

\paragraph{Usage}

\begin{code}[js, %
             caption={Example of a due}, %
             label={lst:due}] %
var my_fn_due = require('due').mock(my_fn); //@\label{lst:due:mock}@

var due = my_fn_due(input);

due.then(function continuation(error, result) { //@\label{lst:due:error-first}@
  if (!error) {
    console.log(result);
  } else {
    throw error;
  }
});
\end{code}


In listing \ref{lst:due}, the function \texttt{my\_fn\_due} synchronously returns a due as a placeholder for its outcome.
The \texttt{then} method of the due allows to define a continuation to continue the execution after retrieving the outcome, like line \ref{lst:due:error-first}.
If the deferred operation is synchronous, the Due settles during its creation and the \texttt{then} method immediately calls this continuation.
If the deferred operation is asynchronous, this continuation is called during the Due settlement.

\vspace{\baselineskip}

\paragraph{Creation} \label{chapter5:due:definition:creation}

\begin{code}[js, %
             caption={Creation of a due}, %
             label={lst:due-creation}] %
Due.mock = function(my_fn) { //@\label{lst:due-creation:mock}@
  return function mocked_fn() { //@\label{lst:due-creation:mocked}@
    var _args = Array.prototype.slice.call(arguments),
        _this = this;

    return new Due(function(settle) {  //@\label{lst:due-creation:new}@
      _args.push(settle);  //@\label{lst:due-creation:push}@
      my_fn.apply(_this, _args); //@\label{lst:due-creation:call}@
    })
  }
}
\end{code}


% A due is often created inside the function which returns it, like in listing \ref{lst:due-creation}.
In listing \ref{lst:due}, line \ref{lst:due:mock}, the \texttt{mock} method wraps the original function \texttt{my\_fn} in a Due-compatible function \texttt{mocked\_fn}.
The \texttt{mock} method is detailed in listing \ref{lst:due-creation} to illustrate the creation of a Due.
% The rest of this code is similar to the Promise example, listing \ref{lst:then}.
It returns a Due compatible function, \texttt{mocked\_fn}, line \ref{lst:due-creation:mocked}.
That is a function that returns a Due, instead of expecting a continuation.

At the execution of \texttt{mocked\_fn} the Due to be returned is created line \ref{lst:due-creation:new}, with the original function passed as argument.
The original function \texttt{my\_fn} is executed during the creation of the Due.
The \texttt{settle} function provided is passed as a continuation line \ref{lst:due-creation:push} for the original function to settle the returned Due. %, synchronously or asynchronously.
% Therefore, the \texttt{settle} function is pushed at the end of the list of arguments, line \ref{lst:due-creation:push}.
% Indeed, the operation might be synchronous, or asynchronous.
% The callback invokes the deferred operation with this list of arguments, and the current context, line \ref{lst:due-creation:call}.
% \texttt{my\_fn} being asynchronous, it expects a callback as last argument : \texttt{settle}.
When the original function completes, it calls \texttt{settle} to settle the Due and save the outcome.
This outcome can then be retrieved with the continuation provided by the \texttt{then} method.
% A Due is in one of two mutually exclusive states: settled or pending.

% This continuation is defined by the \texttt{then} method.
% After the settlement of the Due, its continuation is executed with the outcome.
% Dues expose a \texttt{then} method expecting a continuation to continue the execution after its settlement.


\vspace{\baselineskip}

\paragraph{Composition}

Dues arrange the execution flow as a chain of actions to carry on inputs.
The composition of Dues in a chain is illustrated in listing~\ref{lst:dues-sequence}.
It is similar to the composition of Promises explained in the previous chapter, section~\ref{chapter4:event-loop:promise}, page~\pageref{chapter4:event-loop:promise}.
% explained in section \ref{chapter4:event-loop:promise}.
% It is the same than for Promises,
% This composition is explained in details in section \ref{section:definitions:promise}. %, and illustrated specifically for Dues in listing \ref{lst:dues-sequence}.
% Through this chained composition,

\begin{code}[js, %
             caption={Dues are chained like Promises}, %
             label={lst:dues-sequence}] %
var Due = require('due');

var my_fn_due_1 = Due.mock(my_fn_1),
    my_fn_due_2 = Due.mock(my_fn_2),
    my_fn_due_3 = Due.mock(my_fn_3);

my_fn_due_1(input)
.then(my_fn_due_2) //@\label{lst:due-sequence:then}@
.then(my_fn_due_3)
.then(console.log);
\end{code}

The \texttt{then} method of the current Due returns an intermediary Due that settles when the Due returned by the passed continuation settles.
For example, in listing \ref{lst:dues-sequence} the Due returned by the \texttt{then} method line \ref{lst:due-sequence:then} settles when the Due returned by its continuation \texttt{my\_fn\_due\_2} settles.
It allows to chain continuations one after the other like a pipeline, instead of the nested composition of continuations.

% This simplified specification adopts the same convention than \textit{Node.js} for continuations to hand back outcomes.


% Therefore, the equivalence between a continuation and a Due is trivial.



% This equivalence, and its composition are explained in details in section \ref{section:equivalence}.
% Dues are admittedly tailored for this work, hence, they are not designed to be written by developers, like Promises are.

% The next section presents the equivalence between continuations and Dues.


% In listing \ref{lst:due}, \texttt{due} is settled when the function \texttt{settle} is called.
% If \texttt{due} is settled, a call to \texttt{due.then(onSettlement)} immediately call the function \texttt{onSettlement}.
% A due is pending if it is not settled.
% A due is resolved if it is settled or if it has been linked with another due.
% Attempting to settle a resolved due has no effect.
% A resolved due may be pending or settled, while an unresolved due is always in the pending state.
% The \texttt{Due} object only exposes the \texttt{then} method.
% \textbf{\texttt{Due.prototype.then(onSettlement)}}\\
% Appends settlement handlers to the due, and returns a new due resolving to the return value of the called handler.
% If the value is a \textit{thenable}, \textit{i.e.} has a method \texttt{then}, the returned due will follow that \textit{thenable}, adopting its eventual state; otherwise the returned due will be fulfilled with the value.
% We present in appendix \ref{section:dueimpl} a simple implementation of Due in Javascript.

\subsection{From Continuations to Dues} \label{chapter5:due:equivalence}

The equivalence between continuations and Dues allows the transformation of a nested imbrication of continuations into a chain of Dues.
To preserve the semantic, this transformation imposes limitations on the execution order, the execution linearity and the scopes of the variables used in the operations.

\subsubsection{Execution order}

The transformation of a simple continuation is illustrated in figure \ref{fig:then-transform}
It applies on function calls similar to the abstraction (\ref{eq:order:source}).
It wraps the function $fn$ into the function $fn_\textbf{due}$ to return a Due, as presented in section \ref{chapter5:due:definition:creation}
And it relocates the continuation in a call to the method $\textbf{then}$.
%, which references the Due previously returned.
The result is similar to the abstraction (\ref{eq:order:target}).
The differences are highlighted in bold font.

\begin{figure}[h!]
\centering
\begin{equation} \label{eq:order:source}
fn([arguments], continuation)
\end{equation}
$\downarrow$
\begin{equation} \label{eq:order:target}
fn_\textbf{due}([arguments])\textbf{.then}(continuation)
\end{equation}
\caption{Simple transformation}
\label{fig:then-transform}
\end{figure}

The execution order is different whether $continuation$ is called synchronously, or asynchronously.
If $fn$ is synchronous, it calls the $continuation$ within its execution.
It might execute $statements$ after executing $continuation$, before returning.
If $fn$ is asynchronous, the continuation is called after the end of the current execution, after $fn$.
The transformation erases this difference in the execution order.
In both cases, the transformation relocates the execution of $continuation$ after the execution of $fn$.
For synchronous $fn$, the execution order changes ; the execution of $statements$ at the end of $fn$ and the continuation switch.
To preserve the execution order, the function $fn$ must be asynchronous, or execute the continuation as the last instruction.


\subsubsection{Execution linearity}

The transformation of a chain of continuations into a chain of Dues is illustrated in figure \ref{fig:comp-transform}.
It transforms a nested imbrication of continuations similar to the abstraction (\ref{eq:state:source}) into a flatten chain of calls encapsulating them, as abstraction (\ref{eq:state:target}).

\begin{figure}[h!]
  \begin{minipage}{0.40\textwidth}
    \centering
    \begin{align} \label{eq:state:source}
    &fn1([arguments], cont1 \{\nonumber\\
    &\qquad  declare ~ variable \leftarrow result\nonumber\\
    &\qquad  fn2([arguments], cont2 \{\nonumber\\
    &\qquad\qquad    print ~ variable\nonumber\\
    &\qquad  \})\nonumber\\
    &\})
    \end{align}
  \end{minipage}
  \hfill
  $\to$
  \hfill
  \begin{minipage}{0.40\textwidth}
    \centering
    \begin{align} \label{eq:state:target}
    &\textbf{declare variable}\nonumber\\
    &fn1_\textbf{due}([arguments])\nonumber\\
    &\textbf{.then}(cont1\{\nonumber\\
    &\qquad  variable \leftarrow result\nonumber\\
    &\qquad  \textbf{return} fn2_\textbf{due}([arguments])\nonumber\\
    &\})\nonumber\\
    &\textbf{.then}(cont2\{\nonumber\\
    &\qquad  print ~ variable\nonumber\\
    &\})
    \end{align}
  \end{minipage}
  \caption{Composition transformation}
  \label{fig:comp-transform}
\end{figure}

The main control flow characteristics, is to allow the execution order to be different from the linearity expressed in the source file.
The equivalence takes into account these disruptions when modifying the source.

A call inside a loop yields multiple Dues because of the repetition, while only one can be returned to continue the chain.
The others would be discarded.
% A call nested inside a loop returns multiple Dues, while only one is returned to continue the chain.
% Loops modify the linearity of the execution flow.
% To preserve the semantic, a chain of Dues must not contain any loop.
Similarly, a function definition is not executed in situ.
It breaks the execution linearity, and prevents a call nested within it to return the Due expected to continue the chain in the parent.
% And a call nested inside a function definition is unable to return the Due to continue the chain.
Therefore, the composition transformation doesn't apply on chain of Dues nested inside loops or function definitions.
It must breaks the chain into simple transformation.

On the other hand, conditional branching leaves the semantic intact in a chain of Dues.
If the nested asynchronous function is not called due to branching, the execution of the chain stops as expected. % , either with continuations or Dues.
The transformation to a chain of Dues doesn't impact the semantic.
% We demonstrate the equivalence with a sequence of two continuations.
% The equivalence is sound for a sequence of \textit{n} continuations.

\subsubsection{Variable scope}

In abstraction (\ref{eq:state:source}), the definitions of $cont1$ and $cont2$ are overlapping.
The $variable$ declared in $cont1$ is accessible in $cont2$ to be printed.
In abstraction (\ref{eq:state:target}), however, definitions of $cont1$ and $cont2$ are not overlapping, they are siblings.
The $variable$ is not accessible to $cont2$.
It must be relocated in a parent function to be accessible by both $cont1$ and $cont2$.
The detection of such variables requires to infer their scope statically.
% Languages with a lexical scope define the scope of a variable statically.
Most imperative languages like C/C++, Python, Ruby or Java present a lexical scope, which defines variables scopes statically.
In Javascript, however, the statements \texttt{with} and \texttt{eval} modify the scope dynamically.
The equivalence excludes programs using these statements to keep a lexical scope and be able to infer variable scope statically.


% sThe subset of Javascript excluding the built-in functions \texttt{with} and \texttt{eval} is also lexically scoped.

# Explanation of the concept

## Turn-based programming.







(see presentation on Dues)
-> single-thread, no wait, no block and so on
Shared heap -> no mutex, no synchronization, so it is good scalability


Turn-based programming is an event-loop.
It is the execution of queued events one after the other.
An event is the association of a callback and a message.
The callback is a small Javascript Program, designed to process the message.
During its turn, the callback executes, and can queue events : that is register callback to be executed during a next turn.
TODO what I mean exactly by queue events ? -> the distinction between the asynchronous operation, and the resulting event.

## Pipeline

So a callback sends messages to other callbacks.
-> It is exactly like a pipeline.
However, all the callbacks share the same heap.
So it is not possible to distribute the different callbacks without synchronization of this heap, or splitting the heap for each callback.
TODO state VERY clearly this problem, it is at the core of my thesis.

So, how to split the heap so that each callback has its own exclusive heap ?

## Propagation of variables.

Javascript is lexically scoped, therefore we can identify the scope of variable statically.
(At the exception of eval and with : with is forbidden from strict mode, so that is not a bigdeal, howether, eval is sometimes used in smart ways, but most of the time it is monomorphic (I don't exactly know what that means, I heard from Floreat, it must be something related to PL community)).

### Scope identification

The compiler identifies the variables shared by multiple callbacks from their scope.
TODO explain this in depth.
Function scope, closures, and so on ...

### Scope leaking

Javascript uses a pass-by-sharing paradigm.
That means that sometimes the argument of a call are passed by value, sometimes by reference (atomic data type (number, string, bool) -> by value, complex data type (objects) -> by reference).
That means that the modification of a local variable can affect variable in seemingly unrelated scopes.
It seems that the points-to analysis is what is used to find stuffs like that (side-effects ?).

### Propagation of execution and variables

The execution progress downstream, following the message stream.
TODO state very clearly this proposition, it is the second core of my thesis (and I love the idea, it relates directly to reality, graivity, and the fabric of the universe <3).

Because the propagation of the modification is not instantaneous, going back upstream is like going backward in time : it is impossible.
Therefore, a variable cannot be read upstream a write.
And it cannot be write downstream either.

In other words, only one callback can write on a variable -> seems obvious from previous sections.


In promises, because the heap is not shared, things are less restrictive.
Multiple stages can read and write the same variable, because the propagation of modification is instantaneous, due to the shared heap.
\section{Overall Evaluation} \label{chapter6:evaluation}

The equivalence presented in chapter \ref{chapter4} is implemented in a the fluxional compiler, presented in section \ref{chapter5:flx}.
This implementation is evaluated against the criteria presented in chapter \ref{chapter3}, Productivity, Efficiency and Adoption.

\subsection{Trading Productivity for Efficiency}

% \subsubsection{Productivity}

The equivalence intends to disrupt as less as possible the way developer build web applications.
The goal is to avoid degrading the productivity, hence the adoption, of the proposed platform.
% The source language, Javascript, is left intact, except for the forbidden statements \texttt{with} and \texttt{eval}.
% These statements are already forbidden by some good practice guides \cite{Crockford2008}.
Therefore, the productivity is intended to be the same as the original event-driven platform.

However, in the current state, the compiler implementation is unable to operate the transformation without an external help.
The static analysis is unable to correctly detect the aliasing of the memory in Javascript.
It avoids developers to use Higher-Order Programming, hence impacts composition.
This limitation avoids to improve the current trade-off of productivity for efficiency, as illustrated in table \ref{tab:proposition-productivity}.
Indeed, to gain efficiency, developers need to commit efforts to indicate the stages of the pipeline, and assure their dependency.

% \TablePropositionProductivity{tab:proposition-productivity}

The manual transformation process yields a distributed application, similarly as the other efficient platforms.
And the chapter \ref{chapter3} showed that such applications achieve very good performance efficiency.
But the productivity limitation remains.
It avoids the current implementation to propose a satisfying compromise between productivity and efficiency.
So, the current implementation actually trades productivity for efficiency, similarly to many platform in the state of the art. % , as illustrated in table \ref{tab:proposition-efficiency}.
The perspectives to overcome this limitation are addressed later in section \ref{chapter5:evaluation:perspective}.
% \TablePropositionEfficiency{tab:proposition-efficiency}


% It doesn't make any sense to evaluate an application, as the transformation would not reflect the compilation process, but the manual transformation process.

% If the runtime memory analysis is solid enough to detect correctly the aliasing of the memory, then it will be able to help the development team transitioning from productivity to efficiency, which is the response of this thesis to the problematic.

\subsection{Adoption}

As observed in the chapter \ref{chapter3}, trading productivity for efficiency drastically reduces adoption.
Because the current implementation presents the same limitation than the efficient platforms, its adoption is not expected to be different. %, as illustrated in table \ref{tab:proposition-adoption}.

Yet, both productivity and efficiency are required for the platform to be adopted by new developers as well as large businesses.
Only at this condition, will it reinforce the loop between community and industry.
So the current implementation is not expected to be widely adopted, as presented in the table \ref{tab:proposition-summary}.

\TablePropositionSummary{tab:proposition-summary}
% \TablePropositionAdoption{tab:proposition-adoption}

% It was briefly tested during the development of the grumpy application, presented in chapter \ref{chapter4}, section \ref{chapter4:execution-models:examples}.

The limitation of static analysis avoids the equivalence to be fully implemented to address the problematic.
Hence, this evaluation holds only on the implementation, and not on the equivalence.


When saying that \textit{it is a mistake to attempt high concurrency without help from the compiler}, R. von Behren \textit{et al.} \cite{Behren2003} implies that the language alone cannot achieve high concurrency.
It is necessary to rely on additional tools, such as a compiler to reach the best compromise between productivity and efficiency.
The evaluation of this thesis concludes that static analysis is unable to reach this compromise for the current multi-paradigm languages using higher-order programming.
% Before dropping all higher-order languages for the sake of efficiency,
Yet, there exist alternatives to static analysis to reach this compromise.
The next paragraph presents some interesting perspectives of this work to further address this problematic.

% In the contribution of this thesis, the two main difficulties, identifying stages and detecting memory dependencies, are due to the dynamic nature of Javascript.
% A perspective to overcome these limitation is to implement the transformation, not as a compiler, but as a runtime.
% Indeed, at runtime, all the dynamic behavior are resolved, and the analysis can be much more precise, and less speculative.

% \subsection{Fluxionnal Runtime} 

% \section{Perspectives}

% Javascript is a highly dynamic languages.