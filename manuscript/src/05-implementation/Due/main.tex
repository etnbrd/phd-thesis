\section{Step 1 - Due Compiler} \label{chapter5:due}

% The previous chapter presented globally the state of the art in designing systems to scale in performance, and in maintenance.
% It refined the scope of this thesis to the study of the opposition between maintenance scalability and performance scalability in streaming web applications.
% It concluded with the objectives of this thesis, which is to find an equivalence between the two opposed organizations.
% The maintenance scalability organization, supported by modular programming, higher-order programming and a global memory store.
% The performance scalability organization, supported by the parallelism of memory and execution distribution.
% This section presents the first step in the transformation from the event-driven execution model to the pipeline architecture, as presented in figure \ref{fig:roadmap}.
% That is to identify and extract a pipeline of execution inside an application following the event-driven execution model.
% In this work, we focus on Javascript, and specifically on \textit{Node.js} applications.
%
% Javascript allows higher-order programming.
% It allows to manipulate functions like any object.
% For example to link them to react to asynchronous events, \textit{e.g.} user inputs and remote requests.
% These asynchronously triggered functions are named callbacks, and allow to efficiently cope with the distributed and inherently asynchronous architecture of the Internet.
% To execute a suite of asynchronous functions, callbacks are nested one into the other.
% This nesting, if not organized properly, can result in unreadable layer of callbacks, commonly presented as \textit{callback hell}\ftnt{http://maxogden.github.io/callback-hell/}, or \textit{pyramid of doom}.
%
% Promises are another way to organize a suite of asynchronous operations avoiding this callback hell.
% They organize the operations as a well-defined pipeline.
% Moreover, Promises provide additional control over the asynchronous execution flow, than callbacks.
% They are part of the next version of the Javascript language, ECMAScript 6\ftnt{http://people.mozilla.org/~jorendorff/es6-draft.html}.
% To avoid the equivalence being unnecessarily incomplete, we present an alternative to Promise, called Due.
% Due organize the operations like Promises, as a well-defined pipeline, while discarding the unnecessary additional control over the asynchronous flow.
%

% It focuses on the identification of the chains of causality in continuations.
% Promises bring more control over the asynchronous flow than the chaining of causal sequentiality.
% But they force another control over the execution flow.
% According to the outcome of the operation, they call one function to continue the execution with the result, or another to handle errors.
% This conditional execution is indivisible from the Promise structure.
% As a result, Promises impose a convention on how to hand back the outcome of the deferred computation, while classic continuations leave this conditional execution to the developer.
% To rule out this differences between continuations and Promises, section \ref{chapter5:due} introduces a simpler specification to Promise, called Due.

% This section presents a compiler to identify the pipeline of operating underlying in a Javascript application. % using callbacks, and extract it to express it as Dues.
% The compiler expresses this pipeline as chains of Dues.

% Section \ref{chapter5:due:equivalence} explains the transformation from imbrications of continuations to sequences of Dues.
% Section \ref{chapter5:due:compiler} presents a compiler to automate the application of this equivalence.
% And finally, the developed compiler is evaluated in section \ref{chapter5:due:evaluation}.


% \subsection{From continuations to Promises} \label{seciton:definitions:analysis}

% As detailed in the previous sections, continuations provide the control over the sequentiality of the asynchronous execution flow.
% Promises improve this control to allow chained compositions, and unify the syntax for the synchronous and asynchronous paradigm.
% This chained composition brings a greater clarity and expressiveness to source codes.
% At the light of these insights, it makes sense for a developer to switch from continuations to Promises.
% However, the refactoring of existing code bases might be an operation impossible to carry manually within reasonable time.
% We want to automatically transform an imbrication of continuations into a chained composition of Promises.

% We identify two steps in this transformation.
% The first is to provide an equivalence between a continuation and a Promise.
% The second is the composition of this equivalence.
% Both steps are required to transform imbrications of continuations into chains of Promises.
% to be able to compose this equivalence for imbrications of continuations to obtain chains of Promises.

% Because Promises bring chained composition, the first step might seem trivial as it does not imply any imbrication to transform into chain.
% However, as explained in section \ref{section:definitions:promise}, Promises impose a control over the execution flow that continuations leave free.
% This control induces a common convention to hand back the outcome to the continuation.

% In the Javascript landscape, there is no dominant convention for handing back outcomes to continuations.
% In the browser, many conventions coexist.
% For example, \textit{jQuery}'s \texttt{ajax}\ftnt{http://api.jquery.com/jquery.ajax/} method expects an object with different continuations for success, errors and various other events during the asynchronous operation.
% \textit{Q}\ftnt{http://documentup.com/kriskowal/q/}, a popular library to control the asynchronous flow, exposes two methods to define continuations: \texttt{then} for successes, and \texttt{catch} for errors.
% % The conventions for continuations are very heterogeneous in the browser.
% On the other hand, the \textit{Node.js} API always used the \textit{error-first} convention, encouraging developers to provide libraries using the same convention.
% In this large ecosystem the \textit{error-first} convention is predominant.
% All these examples use different conventions than the Promise specification detailed in section \ref{section:definitions:promise}.
% They present strong semantic differences, despite small syntactic differences.
% % Some conventions include the conditional execution over the outcome, while other conventions let developers provide it.
% % These conventions uses different control-flow.

% To translate these different conventions into the Promises one, the compiler would need to identify them.
% Such an identification might be possible with static analysis methods such as the points-to analysis~\cite{Wei2014}, or a program logic~\cite{Gardner2013,Bodin2014}.
% However, it seems impracticable because of the number and semantical heterogeneity of these conventions.
% Indeed, in the browser, each library seems to provide its own convention.

% In this paper, we are interested in the transformation from imbrications to chains, not from one convention to another.
% The \textit{error-first} convention, used in \textit{Node.js}, is likely to represent a large, coherent code base to test the equivalence.
% Indeed contains currently more than 125 000 packages.
% For this reason, we focus only on the \textit{error-first} convention.
% Thus, our compiler is only able to compile code that follows this convention.
% The convention used by Promises is incompatible.
% We propose an alternative specification to Promise following the \textit{error-first} convention.
% In the next section we present this specification called Due.

% The choice to focus on \textit{Node.js} is also motivated by our intention to compare later the chained sequentiality of Promises with the data-flow paradigm.
% \textit{Node.js} allows to manipulate streams of messages.
% This proved to be efficient for real-time web applications manipulating streams of user requests.
% Both Promises and data-flow arrange the computation in chains of independent operations.
% % In section \ref{section:equivalence}, we explain the two steps of the transformation from continuations to Dues.

% Dues are further defined in section \ref{section:definitions}.

% This made Javascript a language of choice to develop both client and, more recently, server applications for the web.

% Callbacks are well-suited for small interactive scripts.
% But in a complete application, they are ill-suited to control the larger asynchronous execution flow.
% Their use leads to intricate imbrications of function calls and callbacks, commonly presented as \textit{callback hell}\ftnt{http://maxogden.github.io/callback-hell/}, or \textit{pyramid of doom}.
% This is widely recognized as a bad practice and reflects the unsuitability of callbacks in complete applications.
% Eventually, developers enhanced callbacks to meet their needs with the concept of Promise~\cite{Liskov1988}.

% Promises bring a different way to control the asynchronous execution flow, better suited for large applications.
% They fulfill this task well enough to be part of the next version of the Javascript language, ECMAScript 6\ftnt{http://people.mozilla.org/~jorendorff/es6-draft.html}.
% However, because Javascript started as a scripting language, beginners are often not introduced to Promises early enough.
% Most APIs use the classical callback approach encouraging beginner in this practice.
% Moreover, despite its benefits, the concept of Promise is not yet widely acknowledged.
% Developers may implement their own library for asynchronous flow control before discovering existing ones.
% There is such a disparity between the needs for and the adoption of Promises libraries, that there are almost 40 different implementations\ftnt{https://github.com/promises-aplus/promises-spec/blob/master/implementations.md}.

% With the upcoming introduction of Promise as a language feature, we expect an increase of interest, and believe that many developers will shift to this better practice.
% In this paper, we propose a compiler to automate this shift in existing code bases.
% We present the transformation from an imbrication of callbacks to a sequence of Promise operations, while preserving the semantic.

% Promises bring a better way to control the asynchronous execution flow, but they also impose a conditional control over the result of the execution.
% Callbacks, on the other hand, leave this conditional control to the developer.
% This paper focuses on the transformation from imbrication of callbacks to chain of Promises.
% To avoid unnecessary modifications on this conditional control, we introduce an alternative to Promises leaving this conditional control to the developer, like callbacks.
% We call this simpler specification Dues.
% Our approach enables us to compile legacy Javascript code and brings a first automated step toward full Promises integration.
% This simple and pragmatic compiler has been tested over 64 \textit{Node.js} packages from the node package manager (npm\ftnt{https://www.npmjs.com/}), 9 of them with success.

% In section \ref{section:definitions} we define callbacks, Promises and Dues.
% In section \ref{section:equivalence}, we explain the transformation from imbrications of callbacks to sequences of Dues.
% In section \ref{section:compiler}, we present a compiler to automate the application of this equivalence.
% In section \ref{section:evaluation}, we evaluate the developed compiler.

% \subsection{Definition} \label{chapter5:due:definition}

\subsection{Definition} \label{chapter5:due:definition}

A Due is an object used as placeholder for the eventual outcome of a deferred operation.
% Dues are a simplification of the Promise specification.
They are essentially similar to ECMAScript Promises\ftnt{http://www.ecma-international.org/ecma-262/6.0/\#sec-promise-objects}, except for the convention to hand back outcomes.
% They leave the control over the conditional execution over the outcome to the developer.
They use the \textit{error-first} convention, like \textit{Node.js}, as illustrated line \ref{lst:due:error-first} in listing \ref{lst:due}.
The implementation of Dues and its tests are available online\ftnt{https://www.npmjs.com/package/due}.
% A more in-depth description of Dues and their creation follows in the next paragraphs.
% The \texttt{mock} method is implemented in listing \ref{lst:due-creation}.
% While a promise expects two continuations, \texttt{onSuccess} and \texttt{onErrors}, the method \texttt{then} of a due expects only one continuation, following the convention \textit{error-first}.
% \footnotemark{\ref{ftn:error-conventions}}
% \footnotemark{\ref{ftn:error-first}}.

\subsubsection{Usage}

\begin{code}[js, %
             caption={Example of a due}, %
             label={lst:due}] %
var my_fn_due = require('due').mock(my_fn); //@\label{lst:due:mock}@

var due = my_fn_due(input);

due.then(function continuation(error, result) { //@\label{lst:due:error-first}@
  if (!error) {
    console.log(result);
  } else {
    throw error;
  }
});
\end{code}


In listing \ref{lst:due}, the function \texttt{my\_fn\_due} synchronously returns a due as a placeholder for its outcome.
The \texttt{then} method of the due allows to define a continuation to continue the execution after retrieving the outcome, like line \ref{lst:due:error-first}.
If the deferred operation is synchronous, the Due settles during its creation and the \texttt{then} method immediately calls this continuation.
If the deferred operation is asynchronous, this continuation is called during the Due settlement.

\subsubsection{Creation} \label{chapter5:due:definition:creation}

\begin{code}[js, %
             caption={Creation of a due}, %
             label={lst:due-creation}] %
Due.mock = function(my_fn) { //@\label{lst:due-creation:mock}@
  return function mocked_fn() { //@\label{lst:due-creation:mocked}@
    var _args = Array.prototype.slice.call(arguments),
        _this = this;

    return new Due(function(settle) {  //@\label{lst:due-creation:new}@
      _args.push(settle);  //@\label{lst:due-creation:push}@
      my_fn.apply(_this, _args); //@\label{lst:due-creation:call}@
    })
  }
}
\end{code}


% A due is often created inside the function which returns it, like in listing \ref{lst:due-creation}.
In listing \ref{lst:due}, line \ref{lst:due:mock}, the \texttt{mock} method wraps the original function \texttt{my\_fn} in a Due-compatible function \texttt{mocked\_fn}.
The \texttt{mock} method is detailed in listing \ref{lst:due-creation} to illustrate the creation of a Due.
% The rest of this code is similar to the Promise example, listing \ref{lst:then}.
It returns a Due compatible function, \texttt{mocked\_fn}, line \ref{lst:due-creation:mocked}.
That is a function that returns a Due, instead of expecting a continuation.

At the execution of \texttt{mocked\_fn} the Due to be returned is created line \ref{lst:due-creation:new}, with the original function passed as argument.
The original function \texttt{my\_fn} is executed during the creation of the Due.
The \texttt{settle} function provided is passed as a continuation line \ref{lst:due-creation:push} for the original function to settle the returned Due. %, synchronously or asynchronously.
% Therefore, the \texttt{settle} function is pushed at the end of the list of arguments, line \ref{lst:due-creation:push}.
% Indeed, the operation might be synchronous, or asynchronous.
% The callback invokes the deferred operation with this list of arguments, and the current context, line \ref{lst:due-creation:call}.
% \texttt{my\_fn} being asynchronous, it expects a callback as last argument : \texttt{settle}.
When the original function completes, it calls \texttt{settle} to settle the Due and save the outcome.
This outcome can then be retrieved with the continuation provided by the \texttt{then} method.
% A Due is in one of two mutually exclusive states: settled or pending.

% This continuation is defined by the \texttt{then} method.
% After the settlement of the Due, its continuation is executed with the outcome.
% Dues expose a \texttt{then} method expecting a continuation to continue the execution after its settlement.


\subsubsection{Composition}

Dues arrange the execution flow as a chain of actions to carry on inputs.
The composition of Dues in a chain is illustrated in listing~\ref{lst:dues-sequence}.
It is similar to the composition of Promises explained in the previous chapter, section~\ref{chapter4:event-loop:promise}, page~\pageref{chapter4:event-loop:promise}.
% explained in section \ref{chapter4:event-loop:promise}.
% It is the same than for Promises,
% This composition is explained in details in section \ref{section:definitions:promise}. %, and illustrated specifically for Dues in listing \ref{lst:dues-sequence}.
% Through this chained composition,

\begin{code}[js, %
             caption={Dues are chained like Promises}, %
             label={lst:dues-sequence}] %
var Due = require('due');

var my_fn_due_1 = Due.mock(my_fn_1),
    my_fn_due_2 = Due.mock(my_fn_2),
    my_fn_due_3 = Due.mock(my_fn_3);

my_fn_due_1(input)
.then(my_fn_due_2) //@\label{lst:due-sequence:then}@
.then(my_fn_due_3)
.then(console.log);
\end{code}

The \texttt{then} method of the current Due returns an intermediary Due that settles when the Due returned by the passed continuation settles.
For example, in listing \ref{lst:dues-sequence} the Due returned by the \texttt{then} method line \ref{lst:due-sequence:then} settles when the Due returned by its continuation \texttt{my\_fn\_due\_2} settles.
It allows to chain continuations one after the other like a pipeline, instead of the nested composition of continuations.

% This simplified specification adopts the same convention than \textit{Node.js} for continuations to hand back outcomes.


% Therefore, the equivalence between a continuation and a Due is trivial.



% This equivalence, and its composition are explained in details in section \ref{section:equivalence}.
% Dues are admittedly tailored for this work, hence, they are not designed to be written by developers, like Promises are.

% The next section presents the equivalence between continuations and Dues.


% In listing \ref{lst:due}, \texttt{due} is settled when the function \texttt{settle} is called.
% If \texttt{due} is settled, a call to \texttt{due.then(onSettlement)} immediately call the function \texttt{onSettlement}.
% A due is pending if it is not settled.
% A due is resolved if it is settled or if it has been linked with another due.
% Attempting to settle a resolved due has no effect.
% A resolved due may be pending or settled, while an unresolved due is always in the pending state.
% The \texttt{Due} object only exposes the \texttt{then} method.
% \textbf{\texttt{Due.prototype.then(onSettlement)}}\\
% Appends settlement handlers to the due, and returns a new due resolving to the return value of the called handler.
% If the value is a \textit{thenable}, \textit{i.e.} has a method \texttt{then}, the returned due will follow that \textit{thenable}, adopting its eventual state; otherwise the returned due will be fulfilled with the value.
% We present in appendix \ref{section:dueimpl} a simple implementation of Due in Javascript.

\subsection{From Continuations to Dues} \label{chapter5:due:equivalence}

The equivalence between continuations and Dues allows the transformation of a nested imbrication of continuations into a chain of Dues.
To preserve the semantic, this transformation imposes limitations on the execution order, the execution linearity and the scopes of the variables used in the operations.

\subsubsection{Execution order}

The transformation of a simple continuation is illsutrated in figure \ref{fig:then-transform}
The compiler spots function calls similar to the abstraction (\ref{eq:order:source}).
It wraps the function $fn$ into the function $fn_\textbf{due}$ to return a Due, as presented in section \ref{chapter5:due:definition:creation}
And it relocates the continuation in a call to the method $\textbf{then}$.
%, which references the Due previously returned.
The result is similar to the abstraction (\ref{eq:order:target}).
The differences are highlighted in bold font.

\begin{figure}[h!]
\centering
\begin{equation} \label{eq:order:source}
fn([arguments], continuation)
\end{equation}
$\downarrow$
\begin{equation} \label{eq:order:target}
fn_\textbf{due}([arguments])\textbf{.then}(continuation)
\end{equation}
\label{fig:then-transform}
\caption{Simple transformation}
\end{figure}

The execution order is different whether $continuation$ is called synchronously, or asynchronously.
If $fn$ is synchronous, it calls the $continuation$ within its execution.
It might execute $statements$ after executing $continuation$, before returning.
If $fn$ is asynchronous, the continuation is called after the end of the current execution, after $fn$.
The transformation erases this difference in the execution order.
In both cases, the transformation relocates the execution of $continuation$ after the execution of $fn$.
For synchronous $fn$, the execution order changes ; the execution of $statements$ at the end of $fn$ and the continuation switch.
The function $fn$ must be asynchronous to preserve the execution order.

\subsubsection{Execution linearity}

The transformation of a chain of continuations into a chain of Dues is illustrated in figure \ref{fig:comp-transform}.
The compiler transforms a nested imbrication of continuations similar to the abstraction (\ref{eq:state:source}) into a flatten chain of calls encapsulating them, as abstraction (\ref{eq:state:target}).

\begin{figure}[h!]
  \begin{minipage}{0.40\textwidth}
    \centering
    \begin{align} \label{eq:state:source}
    &fn1([arguments], cont1 \{\nonumber\\
    &\qquad  declare ~ variable \leftarrow result\nonumber\\
    &\qquad  fn2([arguments], cont2 \{\nonumber\\
    &\qquad\qquad    print ~ variable\nonumber\\
    &\qquad  \})\nonumber\\
    &\})
    \end{align}
  \end{minipage}
  \hfill
  $\to$
  \hfill
  \begin{minipage}{0.40\textwidth}
    \centering
    \begin{align} \label{eq:state:target}
    &\textbf{declare variable}\nonumber\\
    &fn1_\textbf{due}([arguments])\nonumber\\
    &\textbf{.then}(cont1\{\nonumber\\
    &\qquad  variable \leftarrow result\nonumber\\
    &\qquad  \textbf{return} fn2_\textbf{due}([arguments])\nonumber\\
    &\})\nonumber\\
    &\textbf{.then}(cont2\{\nonumber\\
    &\qquad  print ~ variable\nonumber\\
    &\})
    \end{align}
  \end{minipage}
  \label{fig:comp-transform}
  \caption{Composition transformation}
\end{figure}

Because of the repetition, a call inside a loop yields multiple Dues to chain, while only one is returned to continue the chain.
% A call nested inside a loop returns multiple Dues, while only one is returned to continue the chain.
Loops modify the linearity of the execution flow.
To preserve the semantic, a chain of Dues must not contain any loop.
Similarly, a function definition outside a function call breaks the execution linearity.
This defined function might not be executed synchronously, in the current chain of Dues.
It prevents the nested call to return the Due expected to continue the chain.
% And a call nested inside a function definition is unable to return the Due to continue the chain.
Therefore, the compiler breaks the chain of Dues when loops or function definitions are encountered.

On the other hand, conditional branching leaves the execution linearity and the semantic intact.
If the nested asynchronous function is not called due to branching, the execution of the chain stops as expected, either with continuations or Dues.

% We demonstrate the equivalence with a sequence of two continuations.
% The equivalence is sound for a sequence of \textit{n} continuations.

\subsubsection{Variable scope}

In abstraction (\ref{eq:state:source}), the definitions of $cont1$ and $cont2$ are overlapping.
The $variable$ declared in $cont1$ is accessible in $cont2$ to be printed.
In abstraction (\ref{eq:state:target}), however, definitions of $cont1$ and $cont2$ are not overlapping, they are siblings.
The $variable$ is not accessible to $cont2$.
It must be relocated in a parent function to be accessible by both $cont1$ and $cont2$.
To detect such variables, the compiler must infer their scope statically.
% Languages with a lexical scope define the scope of a variable statically.
Most imperative languages like C/C++, Python, Ruby or Java present a lexical scope, which defines variables scopes statically.
In Javascript, however, the statements \texttt{with} and \texttt{eval} modify the scope dynamically.
The compiler excludes programs using these statements.


% sThe subset of Javascript excluding the built-in functions \texttt{with} and \texttt{eval} is also lexically scoped.

% # Explanation of the concept

% ## Turn-based programming.

% (see presentation on Dues)
% -> single-thread, no wait, no block and so on
% Shared heap -> no mutex, no synchronization, so it is good scalability


% Turn-based programming is an event-loop.
% It is the execution of queued events one after the other.
% An event is the association of a callback and a message.
% The callback is a small Javascript Program, designed to process the message.
% During its turn, the callback executes, and can queue events : that is register callback to be executed during a next turn.
% TODO what I mean exactly by queue events ? -> the distinction between the asynchronous operation, and the resulting event.

% ## Pipeline

% So a callback sends messages to other callbacks.
% -> It is exactly like a pipeline.
% However, all the callbacks share the same heap.
% So it is not possible to distribute the different callbacks without synchronization of this heap, or splitting the heap for each callback.
% TODO state VERY clearly this problem, it is at the core of my thesis.

% So, how to split the heap so that each callback has its own exclusive heap ?

\comment{From here, the reader should be confortable with the event-loop, and the analogy we drawn between the event-loop and a pipeline.
The problematic is now clear : how to split the heap so that each asynchronous callback has its own exclusive heap ?}

\section{Callback identification}

\subsection{\comment{TODO}}

\section{Callback isolation}

We explain in this section the compilation process we developped to isolate the memory access for each callbacks.
The result of this process should be two-fold. First each callback should have an exclusive access on a region of the memory. So that two different callback can be executed in parallel. And it should be clear for each callback, what are the variable needed from upstream callbacks, and what are the variable to send downstream.

\subsection{Propagation of variables}


\subsubsection{Scope identification}

In section \ref{??? Javascript scope / closure}, we explained that Javascript is roughly lexically scoped.
A consequence is that the declaration of contexts can be inferred statically.
For example, in a lexically scoped, strongly typed, compiled language, the compiler know the content of each scope during compile time, and can prepare the memory stack to store the variables in each scope.

In most languages, the memory is in two parts : the stack, and the heap.
The stack is statically scoped, and its layout is known at compile time.
The heap, on the other hand is dynamically allocated. Its layout is built at run time.

But Javascript is a dynamic language, perhaps the most dynamic of all languages.
It doesn't have this distinction between stack and heap. Every variable is dynamically allocated on the heap.
That induce two consequences.
The first is that Javascript provides two statements to dynamically modify the lexial scope : \texttt{eval} and \texttt{with}.
The second is that to know the layout of the heap, we need to use static analysis tools.
In the next two sections, we adress these two consequences.

\subsubsection{Break the lexical scope} \label{???:breakscope}

Without these statements, \texttt{eval} and \texttt{with}, Javascript is lexically scoped. It is possible to infer the scope of each variable at compile time.


The \texttt{with} statement continue the execution using an expression as the lexical scope.
As the provided expression is dynamically evaluated, it is possible to dynamically modify the lexical scope.
The code snippet below show an example of such a situation.

\begin{code}
var aliveCat = {isAlive: true};
var deadCat = {isDead: true}

with (Math.random() > 0.5 ? aliveCat : deadCat) {
  isAlive;
  // Half the time -> ReferenceError: isAlive is not defined
  // Half the time -> true;
}
\end{code}

The variable \texttt{isAlive} is defined only in the object \texttt{aliveCat}.
The presence of the variable \texttt{isAlive} in the lexical environment within the \texttt{with} statement cannot be determined statically, as the lexical environment is dynamically linked to either \texttt{aliveCat} or \texttt{deadCat}.

Note that the MDN reference page on \texttt{with}\ftnt{https://developer.mozilla.org/en-US/docs/Web/JavaScript/Reference/Statements/with} says that \textit{Using \texttt{with} is not recommended, and is forbidden in ECMAScript 5 strict mode.}

The \texttt{}

% TODO and specify that Javascript is roughly lexically scoped : it is not completely lexically scoped, and five examples to backup that.

Not to be mistaken with the \texttt{this} operator.
It is possible to dynamically change the content of an object,  
% TODO continue this paragraph about how the this operator change the properties of an object. 
% Does it change the lexical scope, if that object is actually used as a context elsewhere ? -> No, I don't think so.
% But ask on SO, just to be sure.

\begin{code}
function stuff() {
  this.x = 42;
}

stuff.call({})

\end{code}

% Javascript is lexically scoped, therefore we can identify the the scope of variable statically.
% (At the exception of eval and with : with is forbidden from strict mode, so that is not a bigdeal, howether, eval is sometimes used in smart ways, but most of the time it is monomorphic (I don't exactly know what that means, I heard from Floreat, it must be something related to PL community)).

% The compiler identifies the variables shared by multiple callbacks from their scope.
% TODO explain this in depth.
% Function scope, closures, and so on ...



However, even if Javascript is lexically scoped, the memory is still dynamicall allocated and manipualeted, so that it is not possible to actually infer the memory layout at compiler time only with lexical scope analysis, and without deeper static analysis.

\subsubsection{Scope Leaking}

% Javascript uses a pass-by-sharing paradigm.
% That means that sometimes the argument of a call are passed by value, sometimes by reference (atomic data type (number, string, bool) -> by value, complex data type (objects) -> by reference).
% That means that the modification of a local variable can affect variable in seemingly unrelated scopes.
% It seems that the points-to analysis is what is used to find stuffs like that (side-effects ?).

% TODO what we are talking about here are aliases.

% TODO I am stating here that in low-level language, the memory access is so fine, that it is difficult to exactly pin down the memory layout in term of object, it is rather seen as a big array of memory adresses.
% While in higher-level language, like Javascript, the memory access is at the property level (it is not possible to access memory down to the adress), so it could be easier (maybe, just not harder) to infer the dynamic memory layout from source.
To infer the layout of the heap at compile time, static analysis tools are used, like the points-to analysis, developped by Andersen in its PhD thesis \cite{Andersen1994}.
For such analysis, the memory is splitted at the access scale.
In low-level languages, like C/C++, the memory is mainly managed by the developer. Allowing access to the memory at a small grained scale : up to the address.
It impose the analysis to split the memory to the adress scale in some cases.
% TODO Backup that, HEAVILY
In higher-level languages, like Javascript, the developer cannot access the memory to the adress scale.
The memory is accessed at a coarser scale : the property scale.
(At the exception of some arrays and buffers, that mimic, and are mapped to actual memory adresses for performance reasons.)
% TODO find exactly the references for these buffers : I think of ArrayBuffer, and sharedArray ... but I am not sure. Need more inspection.

\subsubsection{Propagation of execution and variables}

For the execution of each callback / stage, the corresponding part of the state is local, and the rest is remote, and inaccessible.
We are going to explain why it must remain inaccessible.

While a callback is executing a request, the previous callback (the up stream callback) is executing the next request.
The next request will arrive at the current callback some time in the future.
The modification done in the state of the upstream callback will propagate only later in the current callback.
The state of the upstream callback is in a different time frame than the state of the current callback.

To really understand that, we need to compare this execution with the execution on a unique event-loop.
If the current callback executes, then the upstream callback might have, or might not have started to execute the next request.
But as soon as the current callback executes, the modifications done on the states, are immediatly propagated, so that the upstream callback can take them into account for the next request.

However, if the two callbacks are distant, then the modification of the current callback will not immediatly propagate to the upstream callback.
During the propagation, the upstream callback might execute requests than would not be aware of the state modification from the current callback (from downstream).
That is why we say the upstream callback and the current callback are in two different time frame.
Propagating the state modification upstream is like going backward in time, it is impossible.
That is why the execution, and the state modification propagation must always flow downstream.

As a note, I must add that if an upstream and a downstream callbacks are on the same event-loop, then this doesn't apply. it is like a loop in the time : the modification immediatly propagate from downstream to upstream.






% The execution progress downstream, following the message stream.
% TODO state very clearly this proposition, it is the second core of my thesis (and I love the idea, it relates directly to reality, graivity, and the fabric of the universe <3).

% Because the propagation of the modification is not instantaneous, going back upstream is like going backward in time : it is impossible.
% Therefore, a variable cannot be read upstream a write.
% And it cannot be write downstream either.

% In other words, only one callback can write on a variable -> seems obvious from previous sections.


% In promises, because the heap is not shared, things are less restrictive.
% Multiple stages can read and write the same variable, because the propagation of modification is instantaneous, due to the shared heap.





% TODO write about what it implies to detect continuation in variable, or other expressions.

% Why can we only detect continuations declared in situ.
% If a continuation is passed as a variable, we don't know for sure what is the function associated with, and the closure of that function.

\subsection{Real test case} \label{chapter5:flx:evaluation}

The compiler is tested on a real application, gifsockets-server\ftnt{https://github.com/twolfson/gifsockets-server}.
This test proves the possibility for an application to be compiled into a network of independent parts.
It shows the current limitations of this isolation and the modifications needed on the application to circumvent them.

\begin{code}[js, caption={Simplified version of gifsockets-server},label={lst:gifsocket}]
var express = require('express'),
    app = express(),
    routes = require('gifsockets-middleware'), //@\label{lst:gifsocket:gif-mw}@
    getRawBody = require('raw-body');

function bodyParser(limit) { //@\label{lst:gifsocket:bodyParser}@
  return function saveBody(req, res, next) { //@\label{lst:gifsocket:saveBody}@
    getRawBody(req, { //@\label{lst:gifsocket:getRawBody}@
      expected: req.headers['content-length'],
      limit: limit
    }, function (err, buffer) { //@\label{lst:gifsocket:callback}@
      req.body = buffer;
      next(); //@\label{lst:gifsocket:next}@
    });
  };
}

app.post('/image/text', bodyParser(1 * 1024 * 1024), routes.writeTextToImages); //@\label{lst:gifsocket:app.post}@
app.listen(8000);
\end{code}

This application, simplified in listing \ref{lst:gifsocket}, is a real-time chat using gif-based communication channels.
It was selected from the evaluation set of the Due compiler because it is simple enough to illustrate this evaluation.
% \cite{Brodu2015}
%  from the \texttt{npm} registry because it depends on \texttt{express}, it is tested, working, and simple enough to illustrate this evaluation.
The server transforms the received text into a gif frame, and pushes it back to a never-ending gif to be displayed on the client.

On line \ref{lst:gifsocket:app.post}, the application registers two functions to process the requests received on the url \texttt{/image/text}.
The closure \texttt{saveBody}, line \ref{lst:gifsocket:saveBody}, returned by \texttt{bodyParser}, line \ref{lst:gifsocket:bodyParser}, and the method \texttt{routes.write\-Text\-To\-Images} from the external module \texttt{gifsockets-\-middleware}, line \ref{lst:gifsocket:gif-mw}.
The closure \texttt{saveBody} calls the asynchronous function \texttt{getRawBody} to get the request body.
Its callback handles the errors, and calls \texttt{next} to continue processing the request with the next function, \texttt{routes.write\-Text\-To\-Images}.

\subsubsection{Compilation} \label{chapter5:flx:evaluation:compilation}

% We compile this application with the compiler
The compilation result is in listing \ref{lst:flx-gifsocket}.
The function call \texttt{app.post}, line \ref{lst:gifsocket:app.post}, is a rupture point.
However, its callbacks, \texttt{bodyParser} and \texttt{routes.write\-Text\-To\-Images} are not declared \textit{in situ}.
They are evaluated as functions only at runtime.
As precised previously, the compiler discards these callbacks to avoid altering the semantic. % by moving or modifying their definition.
% For this reason, the compiler ignores this rupture point, to avoid interfering with the evaluation.

\begin{code}[flx, caption={Compilation result of gifsockets-server},label={lst:flx-gifsocket}]
flx main & express {req}
>> anonymous_1000 [req, next]
  var express = require('express'),
      app = express(),
      routes = require('gifsockets-middleware'), //@\label{lst:flx-gifsocket:gif-mw}@
      getRawBody = require('raw-body');

  function bodyParser(limit) { //@\label{lst:flx-gifsocket:bodyParser}@
    return function saveBody(req, res, next) { //@\label{lst:flx-gifsocket:saveBody}@
      getRawBody(req, { //@\label{lst:flx-gifsocket:getRawBody}@
        expected: req.headers['content-length'], //@\label{lst:flx-gifsocket:req.headers}@
        limit: limit
      }, >> anonymous_1000 [req, next]);
    };
  }

  app.post('/image/text', bodyParser(1 * 1024 * 1024), routes.writeTextToImages); //@\label{lst:flx-gifsocket:app.post}@
  app.listen(8000);

flx anonymous_1000
-> null
  function (err, buffer) { //@\label{lst:flx-gifsocket:callback}@
    req.body = buffer; //@\label{lst:flx-gifsocket:buffer}@
    next(); //@\label{lst:flx-gifsocket:next}@
  }
\end{code}

The compiler detects a rupture point : the function \texttt{get\-Raw\-Body} and its anonymous callback, line \ref{lst:gifsocket:callback}.
It encapsulates this callback in a fluxion named \texttt{anony\-mous\_\-1000}.
The callback is replaced with a stream placeholder to send the message stream to this downstream fluxion.
The variables \texttt{req} and \texttt{next} are appended to this message stream, to propagate their value from the \texttt{main} fluxion to the \texttt{anony\-mous\_\-1000} fluxion.

When \texttt{anony\-mous\_\-1000} is not isolated from the \texttt{main} fluxion, as if they belong to the same group, the compilation result works as expected.
The variables used in the fluxion, \texttt{req} and \texttt{next}, are still shared between the two fluxions.
In this situation fluxions are quite similar to Dues regarding memory shareing.
Our goal is to isolate the two fluxions, to be able to safely parallelize their executions.

\subsubsection{Isolation} \label{chapter5:flx:evaluation:isolation}

In listing \ref{lst:flx-gifsocket}, the fluxion \texttt{anony\-mous\_\-1000} modifies the object \texttt{req}, line \ref{lst:flx-gifsocket:buffer}, to store the text of the received request, and it calls \texttt{next} to continue the execution, line \ref{lst:flx-gifsocket:next}.
\texttt{req} is an alias to a memory location used in multiple palces in code.
Therefore, these operations produce side-effects that should propagate in the whole application, but the isolation prevents this propagation.
Isolating the fluxion \texttt{anony\-mous\_\-1000} produces runtime exceptions.
The next paragraph details how this situation is handled to allow the application to be parallelized.

\paragraph{Variable \texttt{req}}

The variable \texttt{req} is read in fluxion \texttt{main}, lines \ref{lst:flx-gifsocket:getRawBody} and \ref{lst:flx-gifsocket:req.headers}.
Then its property \texttt{body} is associated to \texttt{buffer} in fluxion \texttt{anony\-mous\_\-1000}, line \ref{lst:flx-gifsocket:buffer}.
The compiler is unable to identify the aliases of this variable. % further usages.
However, the side effect resulting from this association impacts a variable in the scope of the next callback, \texttt{routes.write\-Text\-To\-Images}.
In this test case, the application is modified manually to explicitly propagate this side-effect to the next callback through the function \texttt{next}.
The modifications of this function are explained further in the next paragraph.

\paragraph{Closure \texttt{next}}

The function \texttt{next} is a closure provided by the \texttt{express} \texttt{Router} to continue the execution with the next function to handle the client request.
Because it indirectly relies on the variable \texttt{req}, it is impossible to isolate its execution with the \texttt{anony\-mous\_\-1000} fluxion.
Instead, we modify \texttt{express}, so as to be compatible with the fluxional execution model.
We explain the modifications below.

\begin{code}[flx, caption={Simplified modification on the compiled result},label={lst:mflx-gifsocket}]
flx anonymous_1000
-> express_dispatcher
  function (err, buffer) { //@\label{lst:mflx-gifsocket:callback}@
    req.body = buffer; //@\label{lst:mflx-gifsocket:buffer}@
    next_placeholder(req, -> express_dispatcher); //@\label{lst:mflx-gifsocket:next-placeholder}@
  }

flx express_dispatcher & express {req} //@\label{lst:mflx-gifsocket:express-dispatcher}@
-> null
  function (modified_req) {
    merge(req, modified_req);
    next(); //@\label{lst:mflx-gifsocket:next}@
  }
\end{code}

In listing \ref{lst:gifsocket}, the function \texttt{next} is a continuation allowing the anonymous callback, line \ref{lst:gifsocket:callback}, to call the next function to handle the request.
To isolate the anonymous callback into \texttt{anonymous\_\-1000}, \texttt{next} is replaced by a rupture point.
This replacement is illustrated in listing \ref{lst:mflx-gifsocket}.
The \texttt{express} \texttt{Router} registers a fluxion named \texttt{express\_\-dispatcher}, line \ref{lst:mflx-gifsocket:express-dispatcher}, to continue the execution after the fluxion \texttt{anony\-mous\_\-1000}.
This fluxion is in the same group \texttt{express} as the \texttt{main} fluxion, hence it has access to the original variable \texttt{req}, and to the original function \texttt{next}.
The call to the original \texttt{next} function is replaced by a placeholder to push the stream to the fluxion \texttt{express\_\-dispatcher}, line \ref{lst:mflx-gifsocket:next-placeholder}.
The fluxion \texttt{express\_\-dispatcher} receives the stream from the upstream fluxion \texttt{anony\-mous\_\-1000}, merges back the modification in the variable \texttt{req} to propagate the side effects, and finally calls the original function \texttt{next} to continue the execution, line \ref{lst:mflx-gifsocket:next}.

After the modifications detailed above, the server works as expected.
The isolated fluxion correctly receives, and returns its serialized messages.
The client successfully receives a gif frame containing the text.



\subsection{Limitations}

The static analysis used for this compiler presents some limitations.
It is unable to analyze code with dynamic behaviors.
Higher-order programming leads to more productivity partly beacuse it rely on such dynamic behavior to extend expressivity.
Precisely, it allows more levels of indirections.

\subsubsection{Levels of Indirections}

The indirection is an abstraction between the value, and its manipulation.
In listing \ref{lst:indirection}, the variables \texttt{a} and \texttt{b} point both to the same memory object.
The function \texttt{fn} introduces a level of indirection between the real object \texttt{a} and its manipulation handle, \texttt{b};
% Actually, the variable \texttt{a} already introduces a level of indirection between the real object and the handle \texttt{a}.

\begin{code}[js,
  caption={One level of Indirection},
  label={lst:indirection}]
var a = {
      // an object;
    };

fn(b) {
  // modify b;
}

fn(a);
\end{code}

\subsubsection{Uncertainties}

The indirection is trivial to resolve in listing \ref{lst:indirection}.
It only needs to have access to the definition of \texttt{a} and of \texttt{fn}.
%A very simple static analysis could resolve it.
However, in listing \ref{lst:indirections}, the array \texttt{handlers} introduces a new level of indirection.
The static analysis now needs to have access to the definition of \texttt{i} and of the \texttt{handlers}.
If this definition is provided by an external input, it is not available statically, hence, it adds an uncertainty during the analysis. 

\begin{code}[js,
  caption={Two levels of indirection},
  label={lst:indirections}]
var a = {
      // an object;
    },
    handlers = [
      // definition of fn handlers;
    ],
    i = ?;

handlers[i](a);
handlers[i+1](a);
\end{code}

These examples are extremely simplified.
A real application contains enough indirections for the static analysis to be overwhelmed by uncertainties, and to be unable to resolve the variables.
If a variable is left unresolved, it is impossible to assure its scope and its aliases.
Therefore, the compiler is unable to isolate it into a fluxion, or to distribute its modification by messages.

Moreover, it leads the compiler to ignore the rupture points not defined \textit{in situ}, because their modifications could impact the semantic.
The reason for this precaution, is that the compiler is unable to assure where the function is used, and the scope of its variables.
Therefore, it is unable to assure that the modification will conserve the semantic.

\subsubsection{Dynamic Resolution}

In a web application, this variable \texttt{i} might be part of the user request, which is available only at runtime.
It eventually introduces an uncertainty.

This dynamic resolution of variables is precisely what increase expressiveness.
Trying to resolve them statically is equivalent to restrict expressiveness.
No static analysis can overstep these limitations.
Only a dynamic analysis could analysis the resolved indirections during run time to overstep these limitations correctly.


