\usepackage{marginnote}
\usepackage{xcolor}

\definecolor{todo}{rgb}{0.9,0.5,0.5}
\definecolor{text}{gray}{0.8}
\definecolor{red}{rgb}{1,0,0.29}
\definecolor{gray1}{rgb}{.70,.70,.70}
\definecolor{gray2}{rgb}{.75,.75,.75}
\definecolor{gray3}{rgb}{.80,.80,.80}
\definecolor{gray4}{rgb}{.85,.85,.85}
\definecolor{gray5}{rgb}{.90,.90,.90}
\definecolor{gray6}{rgb}{.95,.95,.95}

\newcommand{\TODO}[1]{%
  % \marginpar
  {
    \textcolor{todo}{\bf TODO}
    \textcolor{text}{#1}
  }
}

\newcommand{\ind }{%
  \hspace{4ex}%
}

\newcommand{\comment}[1]{%
  \textcolor{text}{#1}%
}

\newcommand{\nt}[1]{%
  \textcolor{red}{*}%
  \marginpar{\textcolor{red}{\fontencoding{U}\fontfamily{futs}\selectfont\char 66\relax}\vspace{3mm}\\
  \tiny{\textcolor{text}{#1}}}%
}

\newcommand{\illustration}[1]{%
  \reversemarginpar%
  \marginpar{\tiny{illustration:\\#1}}%
  \normalmarginpar%
}

\newcommand{\ftnt}[1]{%
  \footnote{\small{\url{#1}}}%
}

\newcommand{\cit}[2]{%
  \textit{``#1''}\\[-25pt]%
  \begin{flushright}%
  --- #2%
  \end{flushright}%
}

\newcommand{\V}{ \textcolor{green}{V} }
\newcommand{\X}{ \textcolor{red}{x} }
\newcommand{\U}{ \textcolor{blue}{?} }

\newlength\callStackIndentation
\newcommand{\level}[1]{%
  \setlength\callStackIndentation{2em}%
  \hspace*{#1\callStackIndentation}%
}

\newcommand*\circled[1]{\tikz[baseline=(char.base)]{
            \node[shape=circle,draw,inner sep=0.8pt] (char) {#1};}}

\makeatletter

\tikzstyle{chart}=[
    legend label/.style={font={\scriptsize},anchor=west,align=left},
    legend box/.style={rectangle, draw, minimum size=5pt},
    axis/.style={black,semithick,->},
    axis label/.style={anchor=east,font={\tiny}},
]

\tikzstyle{pie chart}=[
    chart,
    slice/.style={line cap=round, line join=round, very thick,draw=white},
    pie title/.style={font={\bf}},
    slice type/.style 2 args={
        ##1/.style={fill=##2},
        values of ##1/.style={}
    }
]

\tikzstyle{bar chart}=[
    chart,
    bar width/.code={
        \pgfmathparse{##1/2}
        \global\let\bar@w\pgfmathresult
    },
    bar/.style={very thick, draw=white},
    bar label/.style={font={\bf\small},anchor=north},
    bar value/.style={font={\footnotesize}},
    bar width=.75,
]

\pgfdeclarelayer{background}
\pgfdeclarelayer{foreground}
\pgfsetlayers{background,main,foreground}


\newcommand{\pie}[3][]{
    \begin{scope}[#1]
    \pgfmathsetmacro{\curA}{90}
    \pgfmathsetmacro{\r}{0.8}
    \def\c{(0,0)}
    % \node[pie title] at (90:1.3) {#2};
    \foreach \v/\s in{#3}{
        \pgfmathsetmacro{\deltaA}{\v/100*360}
        \pgfmathsetmacro{\nextA}{\curA + \deltaA}
        \pgfmathsetmacro{\midA}{(\curA+\nextA)/2}

        \path[slice,\s] \c
            -- +(\curA:\r)
            arc (\curA:\nextA:\r)
            -- cycle;
        \pgfmathsetmacro{\d}{max((\deltaA * -(.5/50) + 1) , .5)}

        \begin{pgfonlayer}{foreground}
        \path \c -- node[pos=\d,pie values,values of \s]{$\v\%$} +(\midA:\r);
        \end{pgfonlayer}

        \global\let\curA\nextA
    }
    \end{scope}
}

\newcommand{\legend}[2][]{
    \begin{scope}[#1]
    \path
        \foreach \n/\s in {#2}
            {
                  ++(0,-5pt) node[\s,legend box] {} +(3pt,0) node[legend label] {\n}
            }
    ;
    \end{scope}
}