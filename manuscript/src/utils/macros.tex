\usepackage{marginnote}
\usepackage{xcolor}
\usepackage{pifont}

\definecolor{todo}{rgb}{0.9,0.5,0.5}
\definecolor{text}{gray}{0.8}
% \definecolor{red}{rgb}{1,0,0.29}
\definecolor{gray1}{rgb}{.70,.70,.70}
\definecolor{gray2}{rgb}{.75,.75,.75}
\definecolor{gray3}{rgb}{.80,.80,.80}
\definecolor{gray4}{rgb}{.85,.85,.85}
\definecolor{gray5}{rgb}{.90,.90,.90}
\definecolor{gray6}{rgb}{.95,.95,.95}

\definecolor{RED}{RGB}{255, 0, 73}
\definecolor{GREEN}{RGB}{0, 224, 197}
\definecolor{BLUE}{RGB}{57, 117, 178}

\newcommand{\TODO}[1]{%
  % \marginpar
  {
    \textcolor{todo}{\bf TODO}
    \textcolor{text}{#1}
  }
}

\newcommand{\ind }{%
  \hspace{4ex}%
}

\newcommand{\comment}[1]{%
  \textcolor{text}{#1}%
}

\newcommand{\nt}[1]{%
  \textcolor{red}{*}%
  \marginpar{\textcolor{red}{\fontencoding{U}\fontfamily{futs}\selectfont\char 66\relax}\vspace{3mm}\\
  \tiny{\textcolor{text}{#1}}}%
}

\newcommand{\illustration}[1]{%
% \reversemarginpar%
% \marginpar{\tiny{illustration:\\#1}}%
% \normalmarginpar%
}

\newcommand{\ftnt}[1]{%
  \footnote{\small{\url{#1}}}%
}

\newcommand{\cit}[2]{%
  \noindent%
  \textit{{\secfont\fontsize{45pt}{10pt}\selectfont``}\Large{#1}{''}}\\[-5pt]%
  \begin{flushright}%
  --- #2%
  \end{flushright}%
  \vspace{30pt}
}


\newcommand*\rot{\rotatebox{90}}

\newcommand\lab[1]{%
  \rotatebox{90}{\parbox{3cm}{\raggedright #1}}%
}

\newlength\replength
\newcommand\repfrac{.33}
\newcommand\dashfrac[1]{\renewcommand\repfrac{#1}}
\setlength\replength{4.5pt}
\newcommand\rulewidth{1.6pt}

\newcommand\tdotfill[1][\repfrac]{\cleaders\hbox to \replength{%
  \smash{\raisebox{\arraystretch\dimexpr\ht\strutbox-.1ex\relax}{.}}}\hfill}
\newcommand\tabdotline{%
  \makebox[0pt][r]{\makebox[\tabcolsep]{\tdotfill\hfil}}\tdotfill\hfil%
  \makebox[0pt][l]{\makebox[\tabcolsep]{\tdotfill\hfil}}%
  \\[-\arraystretch\dimexpr\ht\strutbox+\dp\strutbox\relax]%
}

\newcommand{\separator}{%
\vspace{20pt}%
\begin{center}%
% \hrulefill%
\begin{tikzpicture}%
\draw (0,0) -- (0.2,0.2);%
\draw (0,0.2) -- (0.2,0);%
\end{tikzpicture}%
% \hrulefill%
\end{center}%
\vspace{10pt}%
}


% \newcommand{\V}{ \textcolor{green}{\ding{51}} }
% \newcommand{\X}{ \textcolor{red}{\ding{53}} }
% \newcommand{\U}{ \textcolor{gray}{?} }
% \newcommand{\J}{ \textcolor{cyan}{\textbf{+}} }
% \newcommand{\M}{ \textcolor{orange}{--} }

\newlength\callStackIndentation
\newcommand{\level}[1]{%
  \setlength\callStackIndentation{2em}%
  \hspace*{#1\callStackIndentation}%
}

\newcommand*{\circled}[1]{\tikz[baseline=(char.base)]{
            \node[shape=circle,draw,inner sep=0.8pt] (char) {#1};}}

\newcommand*{\dotcircled}[1]{\tikz[baseline=(char.base)]{
            \node[shape=circle,draw,line cap=round, dash pattern=on 0pt off 3pt,inner sep=0.8pt] (char) {#1};}}

\newcommand*{\rate}[1]{\tikz[baseline=(char.base)]{%
            \ifnum#1=6 ?%
            \colorlet{tmpcolor}{BLUE}%
            \else%
            \colorlet{tmpcolor}{GREEN!\the\numexpr#1*20!RED}%
            \fi%
            \node[shape=circle,inner sep=0.8pt, fill=tmpcolor] (char) {\textcolor{white}{\ifnum#1=6\textbf?\else\textbf#1\fi}};}}