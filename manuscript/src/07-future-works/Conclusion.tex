\section{Conclusion}

The web allows a new economic model to emerge, and a tremendous number of business opportunities.
To seize these opportunities, a team needs to develop a web application, and grow a business around it.
The economical incentives around the technical development changes completely during the growth of this business.
In the beginning, the development needs to be productive, to quickly release a product, and iterate with the user feedbacks.
While when the project matures, the execution needs to be efficient, to cope with the load of a large user base while limiting the hardware costs.

These two development concerns are incompatible.
No platform can provide both performance efficiency, and development productivity at the same time.
Moreover, the studied platform propose only a fixed compromised between the two.
This work presented a platform to allow the evolution of this compromise to follow the evolution of the economical incentives.

\subsection{Summary}

The following paragraphs presents the contributions of this thesis.

\subsubsection{Fluxional Execution Model} \label{chapter7:conclusion:model}

TODO

\subsubsection{Pipeline Extraction} \label{chapter7:conclusion:extraction}

TODO

\subsubsection{Pipeline Isolation} \label{chapter7:conclusion:isolation}

TODO

\subsection{Opening}

This thesis brought a possible reconciliation between two concerns in the development of a web application, the efficiency of execution and the productivity of development.
However, a third concern is currently increasingly taking importance.
The IT industry have an increasing carbon footprint.
The impact of this lifestyle on the environment starts only to emerge.
It is of crucial importance to limit the carbon emission of our lifestyle.
I leave for future works the reconciliation of the efficient of energy consumption with the two concerns tackled in this thesis.