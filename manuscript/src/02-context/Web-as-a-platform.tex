\section{The Web as a Platform} \label{chapter2:web-as-a-platform}

\begin{wrapfigure}{r}{0.2\textwidth}
  \vspace{-27pt}
  \begin{center}
    \includegraphics[width=0.18\textwidth]{../ressources/Mac-PC.png}
  \end{center}
  \vspace{-20pt}
\end{wrapfigure}

\nt{TODO this paragraph is not clear}

The production and distribution cost for another unit of a software application is virtually null.
The market is only limited by the platform a software can be deployed on.
The bigger the platform, the wider the market.
The Internet and particularly the Web spreads the scalability of software distribution world wide with a near zero latency.
It eventually became the main distribution medium for software, and the wider market there can possibly be.

Similarly to operating systems, Web browsers started as software products with the ability to run scripts to extend their possibilities.
It led the web to become the platform, replacing operating systems.
Now, with web services, or Software as a Service (SaaS), the distribution medium of software is so transparent that owning a software product to have an easier access is no longer relevant.
It stimulates a competly new business model based on a free access for the user, while claiming value for their data.
We explore in the next paragraphs the different languages that allowed this new business model to emerge.

\subsection{Javascript, The Language of the Web}

In the 80's and early 90's, with Moore's law predicting exponential increase in hardware performance, development time became more expensive than hardware.
Higher-level languages replaced lower-level languages.
The economical gain in development time compensated the worsen performances.
During the web early development, most of the now popular programming languages were released, Python(1991), Ruby(1993), Java(1994), PHP(1995) and Javascript(1995).

Java, developed by Sun Microsystems, imposes itself in the software industry and never really decreased.
The language is executed on a virtual machine, allowing to write an application once, and to deploy it on heterogeneous machines.
However, because of the heavy adoption by the software industry, Java lose the hype that drove the community innovation and creativy.
It struggles to keep up with the latest trends in software development.
Similarly, Ruby was almost unknown until the release of the web framework Rails.
Rails was initially created within an industrial context, but is now open source, and supported by a strong community.

Other languages bloomed, grew within a strong community, and were later adopted by the industry for web development.
For example, it is the case of Python and PHP.
It is often said of Python, that \textit{[it] is the second best language for everything} because of number of available libraries.
The Python community brought Django as a web frameworks, and it is used to develop many web applications in industrial contexts.
And PHP was initially designed to build personal web pages.
Its simplicity made it a language of choice for many young developpers.
As an example, Wordpress is the result of this effervescent, and is now a strong economical success.
These examples show that the involvement of the community is critical for the adoption, evolution and maturation of a language.

Since a few years, Javascript is slowly becoming the main language for web development.
It is the only choice in the browser.
Because of this unavoidable position, work has been done to make it fast (V8, ASM.js) and convenient (ES6, ES7).
And since 2009, it is present on the server as well with Node.js
This omnipresence became an advantage.
It allows to develop and maintain the whole application with the same language.

\subsubsection{In the beginning}

Javascript was created by Brendan Eich at Netscape around May 1995, and released to the public in September.
At the time, Java was quickly adopted as the default language for web servers development, and everybody was betting on pushing Java to the client as well.
The history proved them wrong.

% When Javascript was released in 1995, the world wide web was on the rise.\ftnt{http://www.internetlivestats.com/internet-users/}
% Browsers were emerging, and started a battle to show off the best features and user experience to attract the wider public.\footnote{to get an idea of the web in 1997 : \url{http://1x-upon.com/}}
Javascript was released as a scripting engine on Netscape navigator.
Microsoft released their browser Internet Explorer 3 in June 1996 with a concurrent implementation of Javascript.
At the time, because of the differences between the two implementations, web pages had to be designed for a specific browser.
This competition was fragmenting the web.
To stop this fragmentation, Netscape submitted Javascript to Ecma International for standardization in November 1996.
In June 1997, ECMA International released ECMA-262, the first specification of ECMAScript, the standard for Javascript.
A standard to which all browser should refer for their implementations.
% TODO more on the Ecma specification ?

The initial release of Javascript was designed in a rush. The version released in 1995 was finished within 10 days.
And, it was intended to be simple enough to attract unexperienced developers.
For these reasons, the language was considered poorly designed and unattractive by the developer community.

\illustration{the ugly duckling}

{\fontfamily{phv}\fontseries{l}
\fontsize{10pt}{10pt}\selectfont
Why does Javascript suck?\ftnt{http://whydoesitsuck.com/why-does-javascript-suck/}

Is Javascript here to stay?\ftnt{http://www.javaworld.com/article/2077224/learn-java/is-javascript-here-to-stay-.html}

Why Javascript Is Doomed.\ftnt{http://simpleprogrammer.com/2013/05/06/why-javascript-is-doomed/}

Why JavaScript Makes Bad Developers.\ftnt{https://thorprojects.com/blog/Lists/Posts/Post.aspx?ID=1646}

JavaScript: The World's Most Misunderstood Programming Language\ftnt{http://www.crockford.com/javascript/javascript.html}

Why Javascript Still Sucks\ftnt{http://www.boronine.com/2012/12/14/Why-JavaScript-Still-Sucks/}

10 things we hate about JavaScript\ftnt{http://www.infoworld.com/article/2606605/javascript/146732-10-things-we-hate-about-JavaScript.html}

Why do so many people seem to hate Javascript?\ftnt{https://www.quora.com/Why-do-so-many-people-seem-to-hate-JavaScript}
}

But things evolved drastically since.
One of the reason for the success of Javascript is the \textit{View Source} menu that reveals the source code of any web site client side.
% The success of Javascript is due to many factors ; maybe the most important of all is the \textit{View Source} menu that reveals the complete source code of any web application.
% \textit{The view source menu is the ultimate form of open source}\ftnt{http://blog.codinghorror.com/the-power-of-view-source/}.
It allows the community to pick, improve and reproduce the best techniques as an open collaboration \ftnt{http://blog.codinghorror.com/the-power-of-view-source/}.
% It brought open source and collaborative development to the web.
Another reason is that all web browsers include a Javascript interpreter, making Javascript the most ubiquitous runtime in history \cite{Flanagan2006}.
% Every browser include development tools for Javascript, making it the most ubiquitous development environment, as well.
Javascript is distributed freely, with all the tools needed to reproduce and experiment on the largest communication network in history.
Anybody can seize this opportunity to incrementally build and share the best tools they can.
All these reasons made the popularity of the Web and Javascript.

% When such a language is distributed freely with the tools to reproduce and experiment on every piece of code.
% And its distribution is carried during the expansion of the largest communication network in history.
% Then an entire generation seizes this opportunity to incrementally build and share the best tools they can.
% This collaboration is the reason for the popularity of Javascript on the Web.
% When a language is released, available freely at a world wide scale, and simple enough to be handled by a generation of teenager inspired by the technology hype, it produce an effervescent community around what is now one of the most popular and widely used programming language.

\subsubsection{Rising of the unpopular language}

Javascript started as a programming language to implement short interactions on web pages.
The best usage example was to validate some forms on the client before sending the request to the server.
This situation hugely improved since the beginning of the language.
Nowadays, there is a lot of web-based application replacing desktop applications, like mail client, word processor, music player, graphics editor…

% There is now proportionally more software services released to the public as web-based application than desktop clients. \nt{TODO need source}

ECMA International released several version in the few years following the creation of Javascript.
% The first and second version, released in 1997 and 1998, brought minor revisions to the initial draft.
The third version %, released in the late 1999,
contributed to give Javascript a more complete and solid base as a programming language.
From this point on, the consideration for Javascript kept improving.

%An important reason for this reconsideration started in 2005.
In 2005, James Jesse Garrett released \textit{Ajax: A New Approach to Web Applications}, a white paper coining the term Ajax \cite{Garrett2005}.
This paper explains the advantage on user experience of this technique.
% Ajax stands for Asynchronous Javascript And XML.
It uses Javascript to reload the content inside a web page without requesting a full page from the server, but only the new content.
Javascript rose outside of simple user interactions and allows to develop richer applications inside the browser, from user interactions to network communications.
%, while keeping all the advantages of web-based applications.
The first web applications to use Ajax were Gmail, and Google maps\footnote{A more in-depth analysis of the history of Ajax, given by late Aaron Swartz \url{http://www.aaronsw.com/weblog/ajaxhistory}}.

% The third version of ECMAScript had been released, and it was homogeneously supported in the browsers.
% , the DOM, and the \texttt{XMLHttpRequest} method, two components on which ... relies, 
Because of the difference of implementations, AJAX still present heterogeneous interfaces among browsers.
% Around this time, the Javascript community started to emerge.
Around this time, the community realeased Javascript framework with the goal to straighten differences in implementation, and assist the development.
Prototype\ftnt{http://prototypejs.org/} and DOJO\ftnt{https://dojotoolkit.org/} are early famous examples, and later jQuery\ftnt{https://jquery.com/} and underscore\ftnt{http://underscorejs.org/}.
These frameworks took some responsibilities to the large success of Javascript and of the web technologies.

\illustration{Javascript with superpowers}

In the meantime, in 2004, the Web Hypertext Application Technology Working Group\ftnt{https://whatwg.org/} was formed to work on the fifth version of the HTML standard.
The name is misleading, it is really about giving Javascript superpowers.
% This new version provide new capabilities to web browsers, and a better integration with the native environment.
It features geolocation, file API, web storage, canvas drawing element, audio and video capabilities, drag and drop, browser history manipulation, and many mores.
% It gave Javascript the missing interfaces to become the environment required to develop rich application in the browser.
% The first public draft of HTML 5 was released in 2008, and the fifth version of ECMAScript was released in 2009.
The releases of HTML5 and ECMAScript 5, in 2008 and 2009, represent a mile-stone in the development of web-based applications.
Around the same time, Google released V8 for its browser Chrome.
It is a Javascript interpreter improving drastically the execution performance. % with a just-in-time compiler.
Javascript became the programming language of web, becoming the rising application platform.

\nt{TODO More precisions on the following paragraph}

Taking advantage of this trend, Node.js pushed Javascript to the server as well in 2009.
% Javascript was initially proposed to develop user interfaces.
Javascript is often associated with an event-based paradigm to react to concurrent user interactions.
This event-based paradigm proved to be also very efficient for a web server to react to concurrent requests.
Javascript is now the only language able to build a complete web service, from the client to the server.
% I will present this event-based paradigm in the next section.

% Javascript, and web technologies are also used outside the web.
% NW.js\ftnt{https://github.com/nwjs/nw.js} and electron\ftnt{https://github.com/atom/electron} are two solutions to deploy application built with web technologies.
% They use Node.js and Chromium.
% The Atom text editor\ftnt{https://atom.io/}, Popcorn Time\ftnt{https://popcorntime.io/} and Light Table\ftnt{http://lighttable.com/} are example of such applications.
% However, if web applications are common choice for web service client on the desktop, HTML5 is not yet widely accepted as ready to build complete application on mobile, where performance and design are crucial.
% Indeed web-technologies are often not as capable, and well integrated as native technologies.
% But even for native development, Javascript seems to be a language of choice.
% An example is the React Native Framework\ftnt{https://facebook.github.io/react-native/} from Facebook, which allow to use Javascript to develop native mobile applications.
% They prone the philosophy \textit{"learn once, write anywhere"}, in opposition to the usual slogan \textit{"write once, run everywhere"}.\footnote{Used firstly by Sun for Java, but then stolen by many others}
% Another example is Gnome-shell. It uses Javascript to build its interface, and extensions.
% PhoneGap (Cordova) is a huge effort toward bringing web technologies to the mobile.

\nt{Include more about functional language, promises, browser integration, event programming.}

\subsubsection{Event-Loop}

The event-loop is an efficent execution model for concurrent applications on a single processing unit using non-blocking communications.
It relies on a queue storing the received messages before being processed one after the other by the loop.
On each reception, the loop executes a task to process the received message.
% This event is the aggregation of the response, and of a function to continue the execution with the result.
While processing the message, the task can initiate new communications, leading in turn to the queuing of more messages, which trigger more tasks, and so on.
The scheduling of execution is cooperative.
Each task is executed atomically and exclusively, until it yields execution, to continue with the next task in queue.


% The tasks are scheduled cooperatively, and yield execution on asynchronous communication.

In Javascript, these tasks are defined during the communication initiation.
The function called to initiate the execution expects as argument a function to continue the execution at the reception of communication result.
This function is called a callback or a continuation.

In this model, the input stream of data flows through a sequence of callbacks to be processed until the application outputs it.
This stream is never stored, except as a buffer between two callbacks.
On the contrary, The state ramins in memory to be shared by all callbacks.
Because Javascript is of higher-order, the callbacks as well as their execution contexts are part of this state.
They are called closures.
% In Javascript, it includes the closures.

\nt{TODO schema of an event-loop}

% This execution model is similar to the pipeline architecture presented in the next paragraph.


\begin{figure}

\newcommand{\event}[3][10pt]{\chronoevent[markdepth=-#1]{#2}{#3}}
\setupchronoevent{textstyle=\scriptsize,datestyle=\tiny\bf,datesseparation=/,conversionmonth=false}
% \usr\share\texmf-dist\tex\generic\chronosys\chronosyschr.tex:533 +>
% \vspace{-6pt} % I added this line because the space between date and text is too large for scriptsize font.

\startchronology[align=left, startyear=1994,stopyear=2016, height=0pt, startdate=false, stopdate=false, arrow=false, box=true]
%
\chronograduation[event][dateselevation=10pt]{2}


\event[35pt]  {06/2015}     {ECMAScript v6.0}
\event[10pt]  {14/01/2015}  {io.js}
\event[110pt] {06/2011}     {ECMAScript V5.1}
\event[85pt]  {12/2009}     {ECMAScript v5.0}
\event[60pt]  {27/06/2009}  {Node.js}
\event[35pt]  {02/09/2008}  {V8}
\event[10pt]  {22/01/2008}  {HTML5 public draft}
\event[10pt]  {04/06/2004}  {WHATWG}
\event[110pt] {12/1999}     {ECMAScript v3.0}
\event[85pt]  {06/1998}     {ECMAScript v2.0}
\event[60pt]  {06/1997}     {ECMAScript v1.0}
\event[35pt]  {11/1996}     {ECMA Standardization}
\event        {05/1995}     {Javascript Creation}



\stopchronology
\end{figure}

\subsubsection{Current situation}

\cit{When JavaScript was first introduced, I dismissed it as being not worth my attention. Much later, I took another look at it and discovered that hidden in the browser was an excellent programming language.}{Douglas Crockford}

% \cit{JavaScript is the world's most ubiquitous computing runtime.}{John Lam}



% I want to say that Javascript took off because it was carried by the open source community.
% The goal is to introduce the following facts : JS is widely used in the open source community.
% I need to find the argument saying that open source is taking over closed sources : Javascript / open source is taking over Java / closed source.

% TO READ :
% http://www.javaworld.com/article/2077224/learn-java/is-javascript-here-to-stay-.html
% http://blog.codinghorror.com/the-power-of-view-source/
% http://blog.codinghorror.com/javascript-the-lingua-franca-of-the-web/
% http://shaver.off.net/diary/2007/05/10/the-high-cost-of-some-free-tools/


% This success is obvious on the web and in the open source communities.
The rise of Javascript is obvious on the web and the open source communities.
It also seems to be rising in the software industry.
But it is difficult to give an accurate representation of the situation because the software industry often try to keep an edge by keeping a fog of war.
In the following paragraphs, I report some indexes that represent the situation globally, both in the open source community and in the more opaque software industry.
% More detailed informations are available section \ref{appendix:langpop}.

\paragraph{Available resources}

% The TIOBE Programming Community index is a monthly indicator of the popularity of programming languages.
According to the TIOBE Programming Community index, Javascript ranks 8th, as of October 2015, and it was the most rising language in 2014.
This index measure the popularity of a programming language with the number of results on many search engines.
% The results contains the resources and traces of the activity around the language that are used as a measure of the popularity of a programming language.
However, the number of pages doesn't represent the number of readers.
This measure is controversial, and might not be representative.

Alternatively, Javascript ranks 7th on the PYPL, as of October 2015.
The PYPL index is based on Google trends to measure the number of requests on a programming language.
% This index seems to be more accurate, as it depicts the actual interest of the community for a language.
However, it is not representative as it is limited to google searches.

From these indexes, the major programming language is Java, then C/C++, C\# and Python.
These languages are still the most widely taught, and used to write softwares.
% But Javascript is rising to become an important language as well.

\nt{TODO graphical ranking of TIOBE and PYPL}

\paragraph{Developers collaboration platforms}

Online tools of collaboration gives an indicator of the number of developers and project using certain languages.
% A different indicator of the popularity of a language is the number of developers and projects using it.
Javascript is the most used language on \textit{Github}, the most important collaborative development platform, with around 9 millions users.
It represents more than 320 000 repositories, while the second language is Java with more than 220 000 repositories.

% https://github.com/blog/2047-language-trends-on-github
\includegraphics[width=0.9\linewidth]{../../data/js-trends/github-ranks}


Javascript is the language the most cited on \textit{StackOverflow}, the most important Q\&A platform for developers.
It is a good representation of the activity around a language.
Javascript represent more than 960 000 questions, while the second is Java with around 940 000 questions.

\nt{TODO graphical ranking of the tags in StackOverflow}

According to \textit{Black Duck Software}, Javascript is the second language used in open source projects.
C is first, C++ third and Java fourth.
These four languages represent about 80\% of all programming language usage in open source communities.

% Black Duck Software helps companies streamline, safeguard, and manage their use of open source.
% For its activity, it analyzes 1 million repositories over various forges, and collaborative platforms to produce an index of the usage of programming language in open source communities.
% Javascript ranks second.

\nt{TODO redo this graph, it is ugly.}
\includegraphics[width=0.9\linewidth]{../../data/js-trends/black-duck-15}

% \begin{figure}[h!]
% \begin{tikzpicture}
% [
%     pie chart,
%     slice type={c}{gray1},
%     slice type={js}{red},
%     slice type={cpp}{gray2},
%     slice type={java}{gray3},
%     slice type={php}{gray4},
%     slice type={autoconf}{gray5},
%     slice type={python}{gray6},
%     slice type={ruby}{gray1},
%     slice type={xml}{gray2},
%     slice type={sh}{gray3},
%     slice type={asm}{gray4},
%     slice type={sql}{gray5},
%     slice type={make}{gray6},
%     slice type={perl}{gray1},
%     slice type={csharp}{gray2},
%     pie values/.style={font={\small}},
%     scale=2
% ]

% \pie{}{%
%   34.80/c,%
%   15.45/js,%
%   15.13/cpp,%
%   14.02/java,%
%   2.87/php,%
%   2.65/autoconf,%
%   2.15/python,%
%   1.77/ruby,%
%   1.73/xml,%
%   1.18/sh,%
%   1.16/asm,%
%   1.07/sql,%
%   0.94/make,%
%   0.92/perl,%
%   0.90/csharp,%
% }

% \legend[shift={(1.3cm,0.9cm)}]{%
%   {C}/c,%
%   {Javascript}/js,%
%   {C++}/cpp,%
%   {Java}/java,%
%   {PHP}/php,%
%   {Autoconf}/autoconf,%
%   {Python}/python,%
%   {Ruby}/ruby,%
%   {XML Schema}/xml,%
%   {Shell}/sh,%
%   {Assembler}/asm,%
%   {SQL}/sql,%
%   {Make}/make,%
%   {Perl}/perl,%
%   {C\#}/csharp,%
% }
% \end{tikzpicture}
% \caption{Compilation results distribution}
% \end{figure}

\paragraph{Jobs}

The actors of the software industry tends to hide their activities trying to keep on edge on the competition.
All these previous metrics are representing the visible activity about programming language, but are barely representative of the software industry.
The trends on job openings give some hints on the direction the software industry is heading towards.
Javascript is the third most wanted skill, according to \textit{Indeed}, right after SQL and Java.
And over the last 5 years, Javascript almost closed the gap with the first and second position.
% The job searching platform \textit{Indeed} provides some trends over its database of job propositions.
% Javascript developers ranked at the third position, right after SQL and Java developers.
% Over the 5 last years, the number of job position for Javascript developers increased so as to almost close the gap with Java.
Moreover, according to \textit{breaz.io}, Javascript developers get more opportunities than any other developers.
This position indicate that Javascript is increasingly adopted in the software industry.

\nt{TODO redo this graph, it is ugly.}
\includegraphics[width=0.9\linewidth]{../../data/js-trends/jobgraph}

\paragraph{}

All these metrics represent different faces of the current situation of the Javascript adoption in the development community and industry.
It is widely used in open source projects, everywhere on the web, and in the software industry.
With the evolution of web applications development and increased interest in this domain, Javascript is assuredly one of most important language in the times to come.

This section presented the languages used to build web applications.
In the next section presents the realities and technical challenges to assure their performance against billions of users.



\subsection{Highly concurrent web servers}



\subsubsection{Concurrency}

The Internet allows communication at an unprecedented scale.
There is more than 16 billions connected devices, and it is growing fast\ftnt{http://blogs.cisco.com/news/cisco-connections-counter} \cite{Hilbert2011}.
% This massively interconnected network gives the ability for a web applications to be reached at the largest scale.
A large web application like google search receives about 40 000 requests per seconds\ftnt{http://www.internetlivestats.com/google-search-statistics/}.
Such a web application needs to be highly concurrent to manage this amount of simultaneous requests.
% Concurrency is the ability for an application to make progress on several tasks at the same time.
% For example to respond to several simultaneous requests, a task is a part in the response to a request. \nt{TOOD define more clearly what is a task}
%It represents an uninterrupted flow of requests, with a growing throughput.
In the 2000s, the limit to break was 10 thousands simultaneous connections with a single commodity machine\ftnt{http://www.kegel.com/c10k.html}.
Nowadays, in the 2010s, the limit is set at 10 millions simultaneous connections\ftnt{http://c10m.robertgraham.com/p/manifesto.html}.
With the growing number of connected devices on the internet, concurrency is a very important property in the design of web applications.
Moreover, the concurrency needs to be scalable to adapt to this growth of audience, as explained in the next paragraph.

\subsubsection{Scalability}

The traffic of a popular web application such as Google search remains stable because of its popularity.
The importance of the average traffic soften the occasional spikes.
%There is no apparent spikes in the traffic, because of the importance of the average traffic.
However, the traffic of a less popular web application is much more uncertain.
For example, it might become viral when it is efficiently relayed in the media.
% For example, when a web application appears in the evening news, it expects a huge spike in traffic.
% the number of simultaneous requests obviously  increases, and 
The load of the web application increases with the growth of audience.
The available resources needs to increase to meet this load.
This growth can be steady enough to plan the increase of resources ahead of time, or it might be erratic and challenging.
% The spikes of a less popular web application are unpredictable.
% Therefore, the concurrency needs to be expressed in a scalable fashion.
An application is scalable, if it is able to spread over resources proportionnaly as a reaction to the increasing growth of audience.
% For example, if a scalable application uses one resource to handle $n$ simultaneous requests, it will use $k$ resource to handle two times $n$ simultaneous requests.
% With $k$ being constant, for $n$ ranging from tens to millions of simultaneous requests.
% It assures that the load is increasing linearly, instead of exponentially.

\subsubsection{Time-slicing and Parallelism}

Concurrency is achieved differently on hardware with a single or several processing units.
On a single processing unit, the tasks are executed sequentially, interleaved in time, while on several processing units, the tasks are executed simultaneously, in parallel.
% Parallel executions reduce computing time over sequential execution, as it uses more processing units.
If the tasks are independent, they can be executed in parallel as well as sequentially.
This parallelism is scalable, as the independent tasks can stretch the computation on the resources so as to meet the required performance.
% This parallelism is used in operating system to execute several applications concurrently to allow multi-tasking.

However, the tasks within an application need to coordinate together to modify the application state.
This coordination limits the parallelism and impose to execute some tasks sequentially.
It limits the scalability.
The type of possible concurrency, sequential or parallel, is defined by the interdependencies of the tasks.
\nt{TODO the paragraph below is to be rewritten : sentences poorly formed + link with next section}
% Either the tasks are independents and they can be executed in parallel, or the tasks need to coordinate a common state, and they need to be executed sequentially to avoid conflicting accesses to the state.

% The same limitations are found in the two paradigms I study in this thesis : event-loop and multi-process.


\subsubsection{Pipeline}

The pipeline software architecture is composed of isolated stages communicating by message passing to leverage the parallelism of a multi-core hardware architectures.
It is well suited for streaming application, as the stream of data flows from stage to stage.
Each stage has an independent memory to hold its own state.
As the stages are indendent, the state coordination between the stages are communicated along with the stream of data.

\nt{TODO schema of a pipeline}

Each stage is organized in a similar fashion than the event-loop presented above.
It receives and queues messages from upstream stages, processes them one after the other, and outputs the result to downstream stages.
The difference is that in the pipeline architecture, each task is executed on an isolated stage, whereas in the event-loop execution model, all tasks share the same queue, loop and memory store.

Both paradigms encapsulate the execution in tasks assured to have an exclusive access to the memory.
However, they provide two different models to provide this exclusivity resulting in two distinct ways for the developer to assure the invariance on the application state.
Contrary to the pipeline architecture, the event-loop provide a common memory store allowing the best practice of software development to improve maintainability.
It is possibly the reason of the wide adoption of this programming model by the community of developers.

\paragraph{}

This thesis proposes to provide an equivalence between the two memory models for streaming web applications.
The next subsection describes further the similarities and differences between the two models.
The equivalence would allow a compiler to transform an application expressed in one model into the other.
With such a tool, a development team could rely on the common memory store of the event-loop execution model, and focus on the maintainability of the implementation.
And compile continuously during the development the event-loop implementation to the pipeline architecture to assure that the execution can be distributed on a parallel architecture.