\section{Javascript}

\subsection{Explosion of Javascript popularity}

\subsubsection{In the beginning}

Javascript was created by Brendan Eich in 10 days in May 1995, while he was working at Netscape.
The inital name of the project was Mocha, and then switched to LiveScript when released to the public in September 1995.
The name Javascript was later adopter to leverage the trend around Java.
Indeed, Java was considered the new hot web programming language at this time.
% TODO citations : Oreilly Javascript the definitive guide
% TODO read Javascript the good parts

Microsoft released a concurrent implementation of Javascript in June 1996 in their browser Internet Explorer 3.
They changed the name to JScript, to avoid trademark conflict with Oracle Corporation, who owns the name Javascript.
But the differences between the two implementations made difficult for a script to be compatible for both.
Netscape submitted Javascript to Ecma International for standardization in November 1996.
In June 1997, Ecma International released ECMA-262, the first specification of EcmaScript, the standard for Javascript.
% TODO more on the Ecma specification please

It was designed as a simple language to attract unexperienced developers.
By opposition to Java or C++, which target professional developers.
% TODO reformulate that

\subsubsection{Rising of the unpopular language}

Language popularity

Indexes : 
PYPL PopularitY of Programming Languages - Trends on Language + tutorial on google trends
http://pypl.github.io/PYPL.html

TIOBE - trends on Language + programming on a lot of search engine.
http://www.tiobe.com/index.php/content/paperinfo/tpci/index.html
(Javascript is on the wall of fame for 2014, it has the most important rising in the given year)


Stack overflow tags
http://makingdataeasy.com/stackoverflow-trends?tc=relative-button


Github repo
http://adambard.com/blog/top-github-languages-2014/
https://bigquery.cloud.google.com/table/githubarchive:github.timeline?pli=1

\subsubsection{Current situation : complete world domination}

https://www.destroyallsoftware.com/talks/the-birth-and-death-of-javascript

The Atom editor is written in Javascript node.js.



Now, major PaaS (which one) support node.js by default.

Heroku support Python, Java, Ruby, Node.js, PHP, Clojure and Scala

Amazon Lambda Web service support node.js in priority.


>>> News :

npm raises 8m.
http://techcrunch.com/2015/04/14/popular-javascript-package-manager-npm-raises-8m-launches-private-modules/

% >>> I want to say that Javascript is now broadly used.
% Let's just look at the numbers : Javascript is the most popular language on Github, and npm has more package than any other package manager.
% Javascript has the more broadly deployed runtime.
% ... and so on
% >>> the conclusion is : Javascript is now a major language, and it is more than worth the consideration we are giving it in this PhD thesis.

\paragraph{Isomorphic Javascript}

https://www.meteor.com/

https://facebook.github.io/flux/


\paragraph{Reactive}

http://facebook.github.io/react/

Code reuse.
Why it never worked ?


\subsection{Overview of the language}

\subsubsection{Higher-order functions}



\subsubsection{Closures}

A closure is a function associated with the environment in which it was declared.
It is used in languages with higher-order functions, such as Javascript, to have lexical scoping.

\subsection{Turn-based programming}

\subsubsection{the Javascript event-loop}

Web pages are graphical environment offering multiple area of interaction for the user.
Because of this multiplicity, the traditional linear programing model doesn't hold anymore.
Graphical systems switched from this linear programming model to a different programming model focused on events.

Javascript uses higher-order functions.
It is the ability for a language to manipulate functions like any other value.
This ability is used to register a function to trigger after an event occurred.
An event might be the click on an element of the page, for example.

Such a function is named a callback, a handler, a listener ...
And it shift the programming paradigm from synchronous to asynchronous, which is a big deal.

In synchronous programming, the computation step are executed sequentially, one after the other.
The program execution follows perfectly the program layout written in a linear textual file.

On the other hand, asynchronous programming allows a step back from this linearity.


A multi-threaded system allows the developer to explicitly express the parallelism in the application.
A GOTO statement allows the developer to explicitly express the control flow in the application.


Asynchronous programming allows the program to manage the concurrency of the execution.
Unlike a linear layout of an imperative program, it allows to express more finely the dependencies between instructions.

% James Coglan speak about Promises and the abstraction they allow on the control flow. much interesting, especially at the end.
% https://blog.jcoglan.com/2013/03/30/callbacks-are-imperative-promises-are-functional-nodes-biggest-missed-opportunity/

% >>> I want to say that Javascript is the first language broadly used with this asynchronous paradigm.
% Is asynchronous programming a step before declarative programming ?

\subsubsection{Promises}

% Douglas Crowford on Promises
% https://www.youtube.com/watch?v=dkZFtimgAcM

% The web is polluted with dumb, simple Promises tutorial for Javascript
% It is hard to find the relevant information about how Promises change the programmation paradigm.
