\section{An Economical Problem} \label{chapter2:problem-statement}

With the rise of SaaS on the Web, the software industry are in charge of both the development and the execution of the software.
The previous section presented these two aspects individually.
This section presents the challenges encountered by conducting the two at such a large scale.
It then focus on the subject and define the objectives of this thesis.

\subsection{Disrupted Development}

The economical context on the Web allows a project to grow from a very early stage to a large business.
The economical constraints to meet are very different in the beginning and in the maturation of such project.
In the early steps the constraints hold on the development.
The project needs crucially to reduce development costs, and to release a first product as soon as possible.
On the contrary, in the maturation of the project, the constraints hold on the performance.
The product needs to be highly concurrent to meet the load of usage.
The team needs to adapt to meet the different constraints, which implies a disruption in the evolution of the project.
This section further details the reasons and consequences of this disruption.

\subsubsection{Power-Wall Disruption}

\illustration{heating chipset / parallel chipsets}

Around 2004, the speed of sequential execution on a processing unit plateaued\ftnt{https://cartesianproduct.wordpress.com/2013/04/15/the-end-of-dennard-scaling/}.
Manufacturers reached what they called the power wall
They started to arrange transistors into several processing units to keep increasing overall performance while avoiding overheating problems.
Therefore, the performance of the sequential execution plateaued as well.
% required by the cooperative scheduling 
Parallelism is the only option to achieve high concurrency on this parallel hardware.
% Isolating tasks is the only option to achieve high concurrency on this parallel hardware.
But the isolation required by parallelism is in contradiction with the best practices of software development.
This power-wall leads to a rupture between performance and maintainability.

\subsubsection{Unavoidable Modularity}

The best practices in software development advocate to gather features logically into distinct modules.
This modularity allows a developer to understand and contribute to an application one module at a time, instead of understanding the whole application.
It allows to develop and maintain a large code-base by a multitude of developers bringing small, independent contributions.

This modularity avoids a different problem than the isolation required by parallelism.
The former intends to structure code to improve maintainability, while the latter improve performance through parallel execution.
These two organizations are conflicting in the design of the application.
The next paragraph presents the disruptions in the development of a web application implied by this conflict.

\subsubsection{Technological Shift}

Between the prototyping, and the maturation of a web application, the needs are radically different.
During the initiation of a web application project, the economical constraint holds on the pace of development.
The development reactivity is crucial to meet the market needs\ftnt{https://www.cbinsights.com/blog/startup-failure-post-mortem/}.
The development team opt for a popular and accessible language to leverage the advantage of its community.
It is only after a certain threshold of popularity that the  economical constraint on performance requirements exceeds the one on development.
The development team shifts to an organization providing parallelism.

This shift brings two risks.
The development team needs to rewrite the code base to adapt it to a completely different paradigm.
The application risks to fail because of this challenge.
And after this shift the development pace slows down.
The development team cannot react as quickly to user feedbacks to adapt the application to the market needs.
The application risks to fall in obsolescence.

The risks implied by this rupture proves that there is economically a need for a solution that continuously follows the evolution of a web application.
We present in the next section the proposition of this thesis for such a solution.
It would allow developers to iterate continuously on the implementation focusing simultaneously on performance, and on maintainability.

\subsection{Seamless Web Development}

This thesis is conducted in the frame of a larger work within the company Worldline : LiquidIT.
Worldline identified that one of their need was to increase the time to market for its product.
Worldline defines LiquidIT is defined as \textit{a concept of flexible and cost-effective IT services that can be provisioned, built and configured in real time, allowing end-to-end financial transparency}.
It precisely intends to provide \textit{business agility, investment-free charging models, flexibility and ease of use}.
The goal of this thesis in this larger work, is to allow the developer to focus solely on business logic, and leave the technical constraints of deployment to automated tools.
This section presents the objective of this work to avoid the disruption in development, and provide a seamless development experience.
Worldline develops and hosts real-time streaming Web services, as defined in the next paragraph.

\subsubsection{Real-Time Streaming Web Services}

This thesis focuses on web applications processing streams of requests from users in soft real-time.
Such applications receive requests from clients through the HTTP protocol and must respond within a finite window of time.
They are generally organized as sequences of tasks to modify the input stream of requests to produce the output stream of responses.
The stream of requests flows through the tasks, and is not stored.
On the other hand, the state of the application remains in memory to impact the future behaviors of the application.
This state might be shared by several tasks within the application, and imply coordination between them.

The next section introduces the similarities and differences between the two programming models from the previous section.
And then draws an equivalence.
This equivalence is developed throughout this thesis.

\subsubsection{Differences}

Both paradigms encapsulate the execution in tasks assured to have an exclusive access to the memory.
However, they provide two different models to provide this exclusivity resulting in two distinct programming models.
Contrary to the pipeline architecture, the event-loop provides a common memory store allowing the best practice of software development to improve maintainability.

However, these two organizations are incompatible.
Because of economical constraints, this incompatibility implies ruptures in the development.
It represents additional development efforts and important costs.
This thesis argues that it is possible to allow a continuous development between the two organizations, so as to lift these efforts and costs.
This section presents the two programming models representing each an organization.
Then it presents the possibility of an equivalence bridging the two.
This equivalence is detailed further in the the chapter \ref{chapter4} and \ref{chapter5} of this thesis.

\subsubsection{Equivalence}

With this equivalence, it would be possible to express an application following the design principles of software development, hence maintainable.
And yet, the execution engine could adapt itself to any parallelism of the computing machine, from a single core, to a distributed cluster.
Because of the equivalence between these two models, the development team could iterate testing the two models for their different concerns about the implementation : performance and maintainability.
In the beginning of a project, the team focuses on maintainability and evolution, discarding the scalable performance concerns.
And as the project gather audience and the performance concerns become more and more critical, the development team progressively take into account this performance concerns.

The goal of conciliating these two concerns is not new.
The next chapter presents all the results from previous works needed to understand this work, up to the latest results in the field.

\paragraph{}

This thesis proposes to provide an equivalence between the two memory models for streaming web applications.
The next section further describes the similarities and differences between the two models.
The equivalence would allow a compiler to transform an application expressed in one model into the other.
With such a tool, a development team could rely on the common memory store of the event-loop execution model, and focuses on the maintainability of the implementation.
And continuously compiles during the development the event-loop implementation to the pipeline architecture to assure that the execution can be distributed on a parallel architecture.