\section{Summary}

The main contribution of this thesis is the equivalence to allow a continuous shift from productivity to efficiency.
The equivalence map programs written for the event-driven execution model onto the pipeline execution model.
It is based on the identification of rupture point in the synchronous control-flow.
These ruptures points are the boundaries between the stages of the underlying pipeline.
The equivalence extracts, and isolates these stages to execute them in parallel in the fluxional execution model.

% The due compiler transforms the continuation passing style into Promises, to extract the data-flow pipeline underlying in the implementation.
% The fluxional compiler transforms a program written in a productivity language, Javascript, into a network of independent parts of execution.

\subsection{Fluxional Execution Model} \label{chapter7:conclusion:model}

The fluxional execution model executes an application expressed as a network of fluxions.
That is an independent execution container at the function level, communicating by message.
Tthe fluxions are executed in parallel to distribute the computation accross several cores and increase the performance efficiency of the application.

\subsection{Pipeline Extraction} \label{chapter7:conclusion:extraction}

The compiler identifies and extracts the stages of the asynchronous control flow.
% Each asynchronous function acts like a stage in this pipeline.
The first step of this compiler identifies these stages to express them as a pipeline.

The difficulty in this compilation step is to identify the asynchronous functions indicating the stages.
Because of the dynamic nature of Javascript, it is impossible to statically detect these functions.
Hhe compiler is unable to reliably detect them.
Instead the compiler rely on the developer to provide a list of asynchronous function names to extract.

\subsection{Pipeline Isolation} \label{chapter7:conclusion:isolation}

% After the extraction of this pipeline,
The compiler then isolates each stage so that it can be executed in parallel, on the fluxional execution model.

The execution flows from one stage to the other, from the reception of a request until the last stage responds to the request.
The stages are isolated, and communicates only by message along this pipeline.
To communicate the request, and the other state updates, they must follow some rules.
If a stage needs to access variable from one request to the other, this variable is stored in its context.
If a downstream stage needs to read a variable from an upstream stage, the variable is sent as a message in the data-flow of the application.
If two stages needs to share a variable, they are grouped on the same execution node to safely share parts of their context.
Their executions are not parallelized to avoid conflicting accesses.

The difficulty in this step is to identify the memory dependencies between stages.
Because of the dynamic nature of Javascript, there is a lot of possible aliasing in the memory.
The compiler, in its current state is unable to overcome this difficulty.



\section{Economic Considerations}

This thesis intends to bring a reconciliation between two economical concerns in the development of a web application, the efficiency of execution and the productivity of development.
A third concern is currently increasingly taking importance.
The IT industry have an increasing carbon footprint which have an important impact on the environment.
As the IT industry permeates into literally every aspects of our lives, it will be of crucial importance to consider this impact for future developments.
% I leave for future works the reconciliation of the efficiency of energy consumption with the two concerns addresses in this thesis.
% Like with ThinkAir and Maui, whete the code offloading can helps to save energy on the mobile \cite{Kosta2012,Cuervo2010a}.