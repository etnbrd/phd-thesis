\section{Overall Evaluation} \label{chapter6:evaluation}

The equivalence presented in chapter \ref{chapter4} is implemented in a the fluxional compiler, presented in section \ref{chapter5:flx}.
This implementation is evaluated against the criteria presented in chapter \ref{chapter3}, Productivity, Efficiency and Adoption.

\subsection{Trading Productivity for Efficiency}

% \subsubsection{Productivity}

The equivalence intends to disrupt as less as possible the way developer build web applications.
The goal is to avoid degrading the productivity, hence the adoption, of the proposed platform.
% The source language, Javascript, is left intact, except for the forbidden statements \texttt{with} and \texttt{eval}.
% These statements are already forbidden by some good practice guides \cite{Crockford2008}.
Therefore, the productivity is intended to be the same as the original event-driven platform.

However, in the current state, the compiler implementation is unable to operate the transformation without an external help.
The static analysis is unable to correctly detect the aliasing of the memory in Javascript.
It avoids developers to use Higher-Order Programming, hence impacts composition.
This limitation avoids to improve the current trade-off of productivity for efficiency. %, as illustrated in table \ref{tab:proposition-productivity}.
Indeed, to gain efficiency, developers need to commit efforts to indicate the stages of the pipeline, and assure their dependency.

% \TablePropositionProductivity{tab:proposition-productivity}

The manual transformation process yields a distributed application, similarly as the other efficient platforms.
And the chapter \ref{chapter3} showed that such applications achieve very good performance efficiency.
But the productivity limitation remains.
It avoids the current implementation to propose a satisfying compromise between productivity and efficiency.
So, the current implementation actually trades productivity for efficiency, similarly to many platform in the state of the art. % , as illustrated in table \ref{tab:proposition-efficiency}.
The perspectives to overcome this limitation are addressed later in section \ref{chapter6:perspective}.
% \TablePropositionEfficiency{tab:proposition-efficiency}


% It doesn't make any sense to evaluate an application, as the transformation would not reflect the compilation process, but the manual transformation process.

% If the runtime memory analysis is solid enough to detect correctly the aliasing of the memory, then it will be able to help the development team transitioning from productivity to efficiency, which is the response of this thesis to the problematic.

\subsection{Adoption}

As observed in the chapter \ref{chapter3}, trading productivity for efficiency drastically reduces adoption.
Because the current implementation presents the same limitation than the efficient platforms, its adoption is not expected to be different. %, as illustrated in table \ref{tab:proposition-adoption}.

Yet, both productivity and efficiency are required for the platform to be adopted by new developers as well as large businesses.
Only at this condition, will it reinforce the loop between community and industry.
So the current implementation is not expected to be widely adopted, as presented in the table \ref{tab:proposition-summary}.

\TablePropositionSummary{tab:proposition-summary}
% \TablePropositionAdoption{tab:proposition-adoption}

% It was briefly tested during the development of the grumpy application, presented in chapter \ref{chapter4}, section \ref{chapter4:execution-models:examples}.

The limitation of static analysis avoids the equivalence to be fully implemented to address the problematic.
Hence, this evaluation holds only on the implementation, and not on the equivalence.


When saying that \textit{it is a mistake to attempt high concurrency without help from the compiler}, R. von Behren \textit{et al.} \cite{Behren2003} implies that the language alone cannot achieve high concurrency.
It is necessary to rely on additional tools, such as a compiler to reach the best compromise between productivity and efficiency.
The evaluation of this thesis concludes that static analysis is unable to reach this compromise for the current multi-paradigm languages using higher-order programming.
% Before dropping all higher-order languages for the sake of efficiency,
Yet, there exist alternatives to static analysis to reach this compromise.
The next paragraph presents some interesting perspectives of this work to further address this problematic.

% In the contribution of this thesis, the two main difficulties, identifying stages and detecting memory dependencies, are due to the dynamic nature of Javascript.
% A perspective to overcome these limitation is to implement the transformation, not as a compiler, but as a runtime.
% Indeed, at runtime, all the dynamic behavior are resolved, and the analysis can be much more precise, and less speculative.

% \subsection{Fluxionnal Runtime} 

% \section{Perspectives}

% Javascript is a highly dynamic languages.