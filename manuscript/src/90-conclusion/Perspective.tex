\section{Perspectives} \label{chapter6:perspective}

As stated previously, static analysis impacts productivity to favor efficiency.
% For example, to isolate fluxions, the current implementation of the compiler restrains the developer to use only \textit{in situ} callbacks, and avoids aliasing.
Though, an interesting perspective to continue this work is to implement the equivalence as a just-in-time compiler.
Indeed, the dynamic analysis allowed at run time is more prone to overcome the limitation identified with static analysis.

\subsection{Just-in-time Compilation}

Most Javascript interpreters compile some parts of the code at run time to improve performances.
During this compilation, the levels of indirections are mostly resolved.
The code is translated directly into lower-level instructions.

Implementing the equivalence in a just-in-time (JIT) compiler could leverage this dynamic resolution.
It could analyze the scope of variables resolved dynamically, and isolate the stages accordingly.

\paragraph{Rupture point detection}

The asynchronous functions identifying rupture points are not part of Javascript.
They are special functions provided by the interpreter.
With the compiler communicating with the interpreter at run time, detecting rupture points becomes trivial.
The interpreter notifies the compiler when an asynchronous function is called.
The compiler then identifies the rupture point and isolates it to possibly execute it remotely.

\paragraph{Dominator Tree}

To debug the memory in dynamic languages like Javascript, one can use a dominator tree.
It is a tree generated at run time indicating the parenting relations between memory objects.
With such a tree, the analysis of interdependencies between stages becomes trivial.
Each stage can be isolated in a fluxion, and deployed accordingly to its dependencies.

% With the dynamic registering of Fluxions to the messaging system, and into tag groups, it is possible to transform a Javascript application continuously during its execution.
% Analysis of the interdependencies become as trivial as for static languages, with the resolution of the indirections by the just-in-time compiler.
% The fluxional compiler waits for these resolutions, and then analyzes the compiled code for rupture points.
% As the asynchronism of a function call is handled by the execution engine, the just-in-time compilation can pin point precisely the asynchronous calls from the synchronous ones. 
% And the continuations for these asynchronous calls are resolved, which makes them similar to inline continuations.

\paragraph{Closure Serialization}

Closures are required to allow higher-order programming.
But the static compiler is unable to manipulate closures, as illustrated in section \ref{chapter5:flx:evaluation:isolation}.
Closures are generated dynamically by the interpreter.
With the compiler communicating with the interpreter, the former can manipulates and serialize them at run time.
It can then send closures between fluxions, like any other objects.
It enables the use of higher-order programming within the fluxional execution model.
Hence, it would allow, to some extent, to improve the compromise between productivity and efficiency.
Indeed, the developer is free to use the higher-order programming to compose modules, with a global memory abstraction.
Yet, the execution could distribute this global memory abstraction according to the detected interdependencies.

\paragraph{Dynamic Grouping}

With the dynamic detection of stages and their dependencies, and the manipulation of closures, fluxions can be registered during the execution of the application.
To assure they meet their dependencies, the fluxions are deployed according to their groups.
Two fluxions belong to the same group if they need to share access to some variables.
Therefore, they need to be deployed on the same event-loop to share their memory.

\paragraph{Safe-Checking}

It is required to safe-check that the compiled code is consistent with the remaining execution.
As an example, just-in-time compilers check the type to assure that a compiled function remains conform to the input and output types of its call site.
Similarly, it is required to check that the deployment of fluxions doesn't cause inconsistencies.

If a fluxion ready to be deployed belongs to two different groups, these two groups needs to be gathered on the same event-loop.
If they were previously deployed on two different event-loops, they need to be moved with their context to be on the same event-loop.
Moreover, to assure consistency, they need to be moved when receiving the request that triggered the fluxion ready to be deployed.
So that when this new fluxion is executed at this message reception it has access to the contexts of the two groups.
For this purpose, the compiler put the execution on hold, and sends a control message downstream to order the move of the fluxions.
% It interleaves control messages in the stream to communicate with the distributed interpreters.
In this example, the message inquired the distributed interpreters to stop execution, pack the fluxions and their contexts, and send them back to another remote interpreter.
To assure consistency, the execution resumes only when all the fluxions are gathered in the same event-loop, with access to the whole shared memory.

% If the new fluxion depends an a local variable, as well as a variable from a group on another node than the local node, the group needs to be deployed back locally.
% The fluxion as well as all the fluxions of the group are deployed locally, but the execution needs to wait for the contexts of the group to be available locally.
% To gather the contexts, the node responsible for this group send a message to the messaging system managing this group.
% The messaging system gather all the contexts of the fluxions, and send them back.
% When the contexts are deployed locally to the new node responsible for the this group, the execution of this branch can resume.

% A new compromise have to be done between the cost of sending a fluxion and the cost to get it back, and the risk that it requires to be sent back.
% It might be possible to reduce this risk by saving the compilation information from one execution to the other.

\separator

The perspectives described in the previous paragraphs overcome the limitations of the current implementation of the compiler.
They describe the further implementation of the equivalence, as if I were to continue this work.

\subsection{Evaluation of the perspective}

This second evaluation admits that the JIT compilation resolves all the indirections in the memory.
% Which is left unknown after this thesis.
Then, the fluxional JIT compiler doesn't need to rely on human interaction.
Therefore, the expected productivity is the same as the productivity language used as source. %, as illustrated in table \ref{tab:perspective-productivity}.

% \TablePerspectiveProductivity{tab:perspective-productivity}

Naturally, the performance efficiency of the implementation is, at first, the same of this productivity language, as the development is focused on productivity.
Some development efforts are required to improve the efficiency.
% But the result of the compilation helps in this shift.
The result from the compiler helps the developer find the bottle necks, and reduce the effort for this shift.
With the help from the compiler, the effort for this shift is expected to be less than the current required effort.
% The effort for this shift are expected to be less than the current required effort. %, as illustrated in table \ref{tab:perspective-efficiency}.
Instead of redesigning the architecture of the application to immediately isolate components, it is possible to modify them to progressively loosen their dependencies.
As illustrated in table \ref{tab:perspective-summary}, this envisioned platform is expected to yield both productivity and efficiency, not at the same time, but when they are required the most.

% \TablePerspectiveEfficiency{tab:perspective-efficiency}

Moreover, during this decomposition and after, developers can still rely on higher-order programming, even between isolated application parts.
In the current state of the art, there is no known platform to offer higher-order programming between distributed parts.
This possibility is therefore unknown, and could actually yield to an unrivaled compromise between productivity and efficiency.

Following the insight along this thesis, a platform bringing both productivity and efficiency simultaneously would be greatly adopted.
But it requires to be observed in real conditions before drawing this conclusion.
% , as illustrated in table \ref{tab:perspective-adoption}.

% \TablePerspectiveAdoption{tab:perspective-adoption}
% \separator

\TablePerspectiveSummary{tab:perspective-summary}

\subsection{Final Thoughts}

During this thesis, I progressively changed my vision of our everyday world.
As a final note in this thesis, for what it is worth, I would like to share this vision.

Nearly 3 decades ago, the IT industry understood that trading the execution efficiency for development productivity could reduce development time and thus, the final cost.
Hardware performance could compensate the loss of execution efficiency.
But the new challenge to build available web services at a world wide scale requires efficiency again.
Moreover, the IT industry has an important impact on the ecology with its increasing carbon footprint.
As explained all along this thesis, I believe that it is time to take into account both productivity and efficiency.

Yet, programming cannot be reserved to experts anymore.
As digitalization permeates into every aspects of our lives, it is of crucial importance that programming be available for everyone.
Productivity cannot be traded back for efficiency.
This thesis intends to bring a modest reconciliation between two economical concerns for a web application, the efficiency of execution and the productivity of development.
But it still feels like developers are an elite, and are comparable to the copyist monks before the invention of printing.

Yet, some examples show that a shift might actually already be happening.
The platforms Squarespace\ftnt{http://www.squarespace.com/} and the soon to come Grid\ftnt{https://thegrid.io/} allow everyone to launch a store-front on the web without any required programming knowledge.
It illustrates that technology needs to be at the service of social evolution, and not stuck in an ivory tower, as an unreachable field reserved to academics.

Internet disrupted the way we learn and share knowledge.
In this continuity, I believe that in the time to come, programming will be made available for  everybody.
Additionally as reading and writing, programming will be a prerequisite to communicate with peers.
It will allow to express dynamic behaviors, and not only static ones, as Bret Viktor already envisioned\ftnt{https://vimeo.com/115154289}.
I believe that this shift will disrupt the way to see, live and interact with our surroundings and with peers.
A way we could barely imagine today.

This shift of communication might come with the increasing importance of machine learning and artificial intelligence.
Indeed, it allows to easily define complex behaviors, and extend our communication possibilities.
For example, the services of M\ftnt{http://bit.ly/M-facebook}, Jam\ftnt{https://hellojam.fr/}, Magic\ftnt{https://getmagicnow.com/}, and the likes, teams up artificial intelligence with operators to become the new super assistant of everyone.
These autonomous behaviors meld into the human to human interaction to help both parties.
And with the advent of the Internet of things and blockchains smart contracts, it is not far-fetched to imagine our everyday world similarly infused with artificial interactions.
Because of Artificial Intelligence, these complex behaviors will become so common that it will completely dissolve the barrier between human and machine interactions. 

I use the following metaphor to explain my vision on artificial intelligence.
Programming defines precise rules and behaviors to be followed by the machine, similarly to using a chisel to sculpt rocks.
On the other hand, machine learning feels like copying an existing object by molding it.
It injects a neural network into a mold of known behaviors to extend it to unknown behaviors.
It is becoming way easier to program face recognition and natural language processing than to write a kernel module.
Artificial Intelligence might helps developers in:

\begin{itemize}
  \item providing more complex building blocks, and
  \item choosing among the available building blocks.
\end{itemize}

Ultimately, programming is animating matter, by organizing it so as to reproduce a limited part of our consciousness.
I believe that eventually, infused with this consciousness, our surrounding will become autonomous and reactive.
We will no longer interact only with peers but also with our surroundings.
It will become barely recognizable if a behavior is originating from a person, or some machine learning algorithm embodied in our environment.
From this point on, I believe our interactions will meld to form an inextricable mesh.
The distinction between a person and artificial intelligence will dissolve.
And more importantly this distinction will have no importance for us.
It will feel completely natural to ask a car to go fetch groceries.
And to not forget butter, this time.
Similarly to the way our cells gathered to form a greater form of life capable of walking, talking, and thinking, this mesh will be able to do things we cannot even possibly imagine yet.


% \paragraph{Scalability}

% % At the light of this thesis,
% I tend to simplify scalability to the choice of granularity.
% That is to choose which action and information should be kept rather locally or, on the contrary, be spread globally.
% I believe this problematic doesn't only applies to computer system, but to many different everyday organizations, such as economical and social organizations.

% % For example, in economy, it is crucial for certain markets to spread the information worldwide.
% % The variations of prices on international markets yields the informations to organize the consumption and production for the entire planet.
% % It could simply not be organized as efficiently by any other known mean.
% % % It is yet the best way to allocate natural resources.
% % But, the uncontrolled variations of this global economy can be destructive at a smaller scale, and must somehow be contained.
% % Local citizen currencies are an example of such containment.
% % It contains the scope of these variations within a local region.

% % To concludes this thesis, I yield the following problematic.
% Economically, socially, and in many other aspect of our everyday lives, I believe designing an efficient organization boils down to choosing which piece of information shall be kept locally, or spread globally.
% How to layout the organizations composing our everyday world for it to be efficient.

% The organization of the world will start to take wider importance as the development of computing machines permeates the society.

% \paragraph{Accessible And Omnipresent Development}